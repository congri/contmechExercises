\documentclass{article}
\usepackage{nips07submit_e,times}
%\documentstyle[nips07submit_09,times]{article}

\title{\underline{MSE - WS 2016/17}\\
Continuum Mechanics}


\author{
Instructor: P.S. Koutsourelakis  \\
\texttt{p.s.koutsourelakis@tum.de} \\
}

% The \author macro works with any number of authors. There are two commands
% used to separate the names and addresses of multiple authors: \And and \AND.
%
% Using \And between authors leaves it to \LaTeX{} to determine where to break
% the lines. Using \AND forces a linebreak at that point. So, if \LaTeX{}
% puts 3 of 4 authors names on the first line, and the last on the second
% line, try using \AND instead of \And before the third author name.


%
%%%%%%%%%%%%%%%%%%%%%%%%%%%%%%%%%%%%%%%%%%%%%%%%%%%%%%%
\usepackage{amsmath,amssymb}
\usepackage{hyperref} 

\usepackage[dvips]{psfrag,graphicx}


\linespread{1.6}
\newcommand{\ee}{\end{equation}}
\newcommand{\be}{\begin{equation}}
\newcommand{\ec}{\end{center}}
\newcommand{\bc}{\begin{center}}
\newcommand{\eea}{\end{eqnarray}}
\newcommand{\bea}{\begin{eqnarray}}
\newcommand{\bd}{\begin{description}}
\newcommand{\ed}{\end{description}}
\newcommand{\bi}{\begin{itemize}}
\newcommand{\ei}{\end{itemize}}
\newcommand{\pa}{\partial}
\newcommand{\bs}{\boldsymbol}
\def\RR{ \mathbb R}
\newcommand{\refeq}[1]{Equation (\ref{#1})}
%%%%%%%%%%%%%%%%%%%%%%%%%%%%%%%%%%%%%%%%%%%%%%%%%%%%%%
%

\begin{document}

\makeanontitle

%\begin{abstract}
%\input{abstract}
%\end{abstract}

\section*{\"Ubung - Week 01}


\begin{enumerate}
\item (Exam WS2013) Which of the following equations are valid expressions using index notation? Justify your answer!
\bi
\item $\sigma_{ij}=C_{ijkl} \epsilon_{kl}$
\item $a_i b_i = a_j b_j$
\item $a_i b_j = a_j b_i$
\item $\epsilon_{ijk}\epsilon_{ijk}=6$
\item $\epsilon_{ijk} = -\epsilon_{jik}$
\item $\epsilon_{ijk} = \epsilon_{kij}$
\ei

\item Match the meaning of each index notation expression shown below with an option from the list: \\
a) $\lambda =T_{ij}S_{ij}$, \quad b)  $E_{ij}=T_{ik}S_{kj}$, \quad c) $E_{ij}=S_{ki}T_{kj}$, \quad d) 
$a_i=\epsilon_{kij}b_j c_k$, \quad e)  $\lambda=a_ib_i$ \\ f) $\delta_{ij}$, \quad g) $T_{ij}v_j=\lambda v_i$, \quad h)
$a_i=S_{ij}b_j$ , \quad i) $A_{ki}A_{kj}=\delta_{ij}$ , \quad j) $A_{ij}=A_{ji}$ \\
\bi
\item[1] Product of two tensors
\item[2] Product of the transpose of a tensor with another tensor
\item[3] Cross product of two vectors
\item[4] Product of a vector and a tensor
\item[5] Components of the identity tensor
\item[6] Equation for the eigenvalues and eigenvectors of a tensor
\item[7] Double-dot product (scalar product) of tensors
\item[8] Dot product (inner product or scalar product) of two vectors
\item[9] The definition of an orthogonal tensor
\item[10] Definition of a symmetric tensor
\ei

\newpage
\item Compute or simplify:
\bi
\item $x_i \delta_{ij}$
\item $\delta_{ij}\delta_{ij}$
\item $T_{ij}\delta_{jk}\delta_{ki}$
\ei

\item Show that $\delta_{ij} S_{jk} = S_{ik}$

\item If $ T_{ij} =\lambda E_{kk}\delta_{ij}+2\mu E_{ij}$ show that:
\bi
\item $T_{ij} E_{ij}=2\mu E_{ij}E_{ij}+\lambda (E_{kk})^2$
\item $T_{ij} T_{ij}= 4\mu ^2 E_{ij} E_{ij}+ (E_{kk})^2 (4\mu \lambda+3\lambda^2)$ 
\ei


\item If $S_{ij}=T_{ij} - \frac{1}{3}T_{kk} \delta_{ij}$, calculate $S_{ii}$ (a tensor with this property is called
deviatoric, and $\bs S$ is called the deviatoric part of $\bs T$.)


\item  Show that:
\bi
\item  If $T_{ij}=-T_{ji}$ , then $T_{ij} a_i a_j=0$
\item If $T_{ij}=-T_{ji}$  and $S_{ij} =S_{ji}$ , then $T_{ij} S_{ij}=0$
\ei

\item Obtain the matrix for the $2^{\textrm{nd}}$ order tensor $\bs{T}$ that transforms the base vectors as follows: \\
$\bs{T}\bs{e}_1=2\bs{e}_1+\bs{e}_3$, $\bs{T}\bs{e}_2=\bs{e}_2+3\bs{e}_3$, $\bs{T}\bs{e}_3=-\bs{e}_1+3\bs{e}_2$.

\item  Let $\bs{T}$ and $\bs{T}'$ be the matrices of the same $2^{\textrm{nd}}$ order tensor in two different coordinate
systems related by an orthogonal transformation matrix $\bs Q$. Show that $\det(\bs{T}) = \det(\bs{T}')$.


\item  Let $\bs{F}$ be an arbitrary tensor.
\bi
\item Show that $\bs{F}\bs{F}^T$ and $\bs{F}^T\bs{F}$ are both symmetric tensors
\item  If $\bs{F}=\bs{QU}=\bs{VQ}$ where $\bs{Q}$ is orthogonal and $\bs{U,V}$ are symmetric, show that
$\bs{U}^2=\bs{F}^T\bs{F}$ and $\bs{V}^2=\bs{F}\bs{F}^T$.
\ei

\item  Let $\{\bs{\hat{e}}_i\}_{i = 1}^3$  and $\{\bs{\hat{e}}'_i\}_{i = 1}^3$ be two 3-dimensional orthonormal
bases. If $\bs{\hat{e}}_i' = Q_{mi}\bs{\hat{e}}_m$,
\bi
\item  Verify $Q_{mi}Q_{mj}=Q_{im}Q_{jm}=\delta_{ij}$
\item  Show that $\bs{\hat{e}}_i=Q_{im}\bs{\hat{e}}'_m$
\ei

\end{enumerate}



\end{document}
