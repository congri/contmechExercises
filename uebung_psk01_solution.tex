\documentclass{article}
\usepackage{nips07submit_e,times}
%\documentstyle[nips07submit_09,times]{article}

\title{\underline{MSE - WS 2016/17}\\
Continuum Mechanics}


\author{
Instructor: P.S. Koutsourelakis  \\
\texttt{p.s.koutsourelakis@tum.de} \\
}

% The \author macro works with any number of authors. There are two commands
% used to separate the names and addresses of multiple authors: \And and \AND.
%
% Using \And between authors leaves it to \LaTeX{} to determine where to break
% the lines. Using \AND forces a linebreak at that point. So, if \LaTeX{}
% puts 3 of 4 authors names on the first line, and the last on the second
% line, try using \AND instead of \And before the third author name.


%
%%%%%%%%%%%%%%%%%%%%%%%%%%%%%%%%%%%%%%%%%%%%%%%%%%%%%%%
\usepackage{amsmath,amssymb}
\usepackage{hyperref} 
\usepackage{paralist} %enumerate with a) b) c) etc.

\usepackage[dvips]{psfrag,graphicx}


\linespread{1.6}
\newcommand{\ee}{\end{equation}}
\newcommand{\be}{\begin{equation}}
\newcommand{\ec}{\end{center}}
\newcommand{\bc}{\begin{center}}
\newcommand{\eea}{\end{eqnarray}}
\newcommand{\bea}{\begin{eqnarray}}
\newcommand{\bd}{\begin{description}}
\newcommand{\ed}{\end{description}}
\newcommand{\bi}{\begin{itemize}}
\newcommand{\ei}{\end{itemize}}
\newcommand{\pa}{\partial}
\newcommand{\bs}{\boldsymbol}
\def\RR{ \mathbb R}
\newcommand{\refeq}[1]{Equation (\ref{#1})}
%%%%%%%%%%%%%%%%%%%%%%%%%%%%%%%%%%%%%%%%%%%%%%%%%%%%%%
%

\begin{document}

\makeanontitle

%\begin{abstract}
%\input{abstract}
%\end{abstract}

\section*{\"Ubung - Week 01}


\begin{enumerate}
\item (Exam WS2013) Which of the following equations are valid expressions using index notation? Justify your answer!
\bi
\item $\sigma_{ij}=C_{ijkl} \epsilon_{kl}$
\item $a_i b_i = a_j b_j$
\item $a_i b_j = a_j b_i$
\item $\epsilon_{ijk}\epsilon_{ijk}=6$
\item $\epsilon_{ijk} = -\epsilon_{jik}$
\item $\epsilon_{ijk} = \epsilon_{kij}$
\ei

\underline{Solution:}
\bi
\item \makebox[3cm]{$\sigma_{ij}=C_{ijkl} \epsilon_{kl}$\hfill} \textit{valid}
\item \makebox[3cm]{$a_i b_i = a_j b_j$\hfill} \textit{valid: relabeling of dummy indices}
\item \makebox[3cm]{$a_i b_j = a_j b_i$\hfill} \textit{invalid}
\item \makebox[3cm]{$\epsilon_{ijk}\epsilon_{ijk}=6$\hfill} \textit{valid}
\item \makebox[3cm]{$\epsilon_{ijk} = -\epsilon_{jik}$\hfill} \textit{valid: $\bs \epsilon$ antisymmetric under
permutation}
\item \makebox[3cm]{$\epsilon_{ijk} = \epsilon_{kij}$\hfill} \textit{valid: $\epsilon_{ijk} = - \epsilon_{ikj} =
(-1)^2 \epsilon_{kij} = \epsilon_{kij}$}
\ei

\newpage
\item Match the meaning of each index notation expression shown below with an option from the list: \\
a) $\lambda =T_{ij}S_{ij}$, \quad b)  $E_{ij}=T_{ik}S_{kj}$, \quad c) $E_{ij}=S_{ki}T_{kj}$, \quad d) 
$a_i=\epsilon_{kij}b_j c_k$, \quad e)  $\lambda=a_ib_i$ \\
f) $\delta_{ij}$, \quad g) $T_{ij}v_j=\lambda v_i$, \quad h) $a_i=S_{ij}b_j$ , \quad i) $A_{ki}A_{kj}=\delta_{ij}$ ,
\quad j) $A_{ij}=A_{ji}$ \\
\bi
\item[1] Product of two tensors
\item[2] Product of the transpose of a tensor with another tensor
\item[3] Cross product of two vectors
\item[4] Product of a vector and a tensor
\item[5] Components of the identity tensor
\item[6] Equation for the eigenvalues and eigenvectors of a tensor
\item[7] Double-dot product (scalar product) of tensors
\item[8] Dot product (inner product or scalar product) of two vectors
\item[9] The definition of an orthogonal tensor
\item[10] Definition of a symmetric tensor
\ei
\underline{Solution:}
\begin{compactenum}[a)]
  \item \makebox[3cm]{$\lambda =T_{ij}S_{ij}$\hfill} \textit{Double-dot product (scalar product) of tensors}
  \item \makebox[3cm]{$E_{ij}=T_{ik}S_{kj}$\hfill} \textit{Product of two tensors}
  \item \makebox[3cm]{$E_{ij}=S_{ki}T_{kj}$\hfill} \textit{Product of the transpose of a tensor with another tensor}
  \item \makebox[3cm]{$a_i=\epsilon_{kij}b_j c_k$\hfill} \textit{Cross product of two vectors}
  \item \makebox[3cm]{$\lambda=a_ib_i$\hfill} \textit{Dot product (inner product or scalar product) of two vectors}
  \item \makebox[3cm]{$\delta_{ij}$\hfill} \textit{Components of the identity tensor}
  \item \makebox[3cm]{$T_{ij}v_j=\lambda v_i$\hfill} \textit{Equation for the eigenvalues and eigenvectors of a tensor}
  \item \makebox[3cm]{$a_i=S_{ij}b_j$\hfill} \textit{Product of a vector and a tensor}
  \item \makebox[3cm]{$A_{ki}A_{kj}=\delta_{ij}$\hfill} \textit{The definition of an orthogonal tensor}
  \item \makebox[3cm]{$A_{ij}=A_{ji}$\hfill} \textit{Definition of a symmetric tensor}
\end{compactenum}

\newpage
\item Compute or simplify:
\bi
\item $x_i \delta_{ij}$
\item $\delta_{ij}\delta_{ij}$
\item $T_{ij}\delta_{jk}\delta_{ki}$
\ei
%

\underline{Solution:} \\
 Summation convention: indices appearing twice are summed. Kronecker delta $\delta_{ij} = 1$ if $i = j$ and
 $\delta_{ij} = 0$ else.
\bi
\item $x_i \delta_{ij} = \sum_{i = 1}^3 x_i \delta_{ij} = \sum_{i = 1, i \neq j}^3 (0\cdot x_i) + 1\cdot x_j = x_j$
\item $\delta_{ij}\delta_{ij}=\sum_{i=1}^3 \sum_{j=1}^3 \delta_{ij}\delta_{ij}=\sum_{i=1}^3 \delta_{ii}=\delta_{ii}=\delta_{11}+\delta_{22}+\delta_{33}=3$.
\item Similarly
\begin{align}
T_{ij}\delta_{jk}\delta_{ki} &= \sum_{i=1}^3\sum_{j=1}^3 \sum_{k=1}^3 T_{ij}\delta_{jk}\delta_{ki}=\sum_{i=1}^3\sum_{j=1}^3  T_{ij}\delta_{ji} = \sum_{i=1}^3
T_{ii} \\ \nonumber
&= T_{ii} = \textrm{Tr}(\bs T) = T_{11}+T_{22}+T_{33}.
\end{align}
\ei

\item Show that $\delta_{ij} S_{jk} = S_{ik}$

\underline{Solution:}\\
\begin{align}
\delta_{ij} S_{jk} = \sum_{j = 1}^3 \delta_{ij}S_{jk} = \sum_{j = 1, j\neq i}^3(0\cdot S_{jk}) + 1\cdot S_{ik} =
S_{ik}
\end{align}

%
\item If $ T_{ij} =\lambda E_{kk}\delta_{ij}+2\mu E_{ij}$ show that:
\bi
\item $T_{ij} E_{ij}=2\mu E_{ij}E_{ij}+\lambda (E_{kk})^2$
\item $T_{ij} T_{ij}= 4\mu ^2 E_{ij} E_{ij}+ (E_{kk})^2 (4\mu \lambda+3\lambda^2)$ 
\ei
%
\underline{Solution:} \\
\bi
\item By substitution and employment of the summation convention, we obtain:
\be
\begin{array}{ll}
 T_{ij} E_{ij} &=(\lambda E_{kk}\delta_{ij}+2\mu E_{ij})E_{ij} \\
 &= \lambda E_{kk} \delta_{ij}E_{ij}+2\mu E_{ij} E_{ij} \\
 &= \lambda E_{kk} E_{ii}+ 2\mu E_{ij} E_{ij} \\
 &=\lambda(E_{kk})^2+ 2\mu E_{ij} E_{ij}.
\end{array}
\ee
\item Also (change naming of dummy indices to make sure that they appear only twice)
\be
\begin{array}{ll}
 T_{ij} T_{ij} &=(\lambda E_{kk}\delta_{ij}+2\mu E_{ij})(\lambda E_{ll}\delta_{ij}+2 \mu E_{ij}) \\
 &= \lambda^2 E_{kk} E_{ll} \delta_{ij}\delta_{ij}+\lambda E_{kk}\delta_{ij}2 \mu E_{ij}+
 2\mu E_{ij}\lambda E_{ll}\delta_{ij}+4\mu^2 E_{ij} E_{ij} \\
  &= 3 \lambda^2 E_{kk} E_{ll} + 2\lambda \mu E_{kk} E_{ii} +2\lambda \mu E_{ll} E_{ii}+ 4\mu^2 E_{ij} E_{ij}  \\
 &=4\mu ^2 E_{ij} E_{ij}+ (E_{kk})^2 (4\mu \lambda+3\lambda^2).
\end{array}
\ee
\ei

\item If $S_{ij}=T_{ij}-\frac{1}{3}T_{kk} \delta_{ij}$, calculate $S_{kk}$ (a tensor with this property is called deviatoric, and $S$ is 
called the deviatoric part of $T$).


\underline{Solution:}
\be
S_{ii}=T_{ii}-\frac{1}{3}T_{kk} \delta_{ii}=T_{ii}-\frac{1}{3}T_{kk} 3=T_{ii}-T_{kk}=0.
\ee

%
\item  Show that:
\bi
\item  If $T_{ij}=-T_{ji}$ , then $T_{ij} a_i a_j=0$
\item If $T_{ij}=-T_{ji}$  and $S_{ij} =S_{ji}$ , then $T_{ij} S_{ij}=0$
\ei

\underline{Solution:}
\bi
\item Note that since $i,j$ are dummy indices, they can be relabeled:
\begin{align}
T_{ij} a_i a_j &= T_{ji} a_j a_i && \textit{relabeling: $i$ is called $j$, $j$ is called $i$} \\ \nonumber
&= -T_{ij}a_ja_i && \textit{use antisymmetry of ~$\bs T$} \\ \nonumber
&=-T_{ij}a_ia_j && \textit{commutation of $a_i,a_j$} \\ \nonumber
& = 0.
\end{align}
\item  Note that since $i,j$ are dummy indices they can be relabeled:
\begin{align}
 T_{ij} S_{ij} &= T_{ji}S_{ji} && \textit{relabeling: $i$ is called $j$, $j$ is called $i$} \\ \nonumber
 &=-T_{ij} S_{ij} && \textit{use antisymmetry of ~$\bs T$ and symmetry of ~$\bs S$} \\ \nonumber
 &= 0.
\end{align}
\ei


\item Obtain the matrix for the tensor $\bs{T}$, that transforms the base vectors as follows: \\
$\bs{T}\bs{e}_1=2\bs{e}_1+\bs{e}_3$, $\bs{T}\bs{e}_2=\bs{e}_2+3\bs{e}_3$, $\bs{T}\bs{e}_3=-\bs{e}_1+3\bs{e}_2$.
%

\underline{Solution:}
\be
T_{ij} = \bs e_i \cdot \bs T \bs e_j,
\ee
hence
\be
[\bs{T}]=\left[ \begin{array}{lll} 2 & 0 & -1 \\ 0 & 1& 3 \\ 1 &3 & 0 \end{array} \right]
\ee
(see also solved examples 2.7.1, 2.7.2, 2.7.3 in Lai et al.)

\item  Let $\bs{T}$ and $\bs{T}'$ be the matrices of the same $2^{\textrm{nd}}$ order tensor in two different coordinate
systems related by an orthogonal transformation matrix $\bs Q$. Show that $\det(\bs{T}) = \det(\bs{T}')$.

\underline{Solution:}

We have established that $\bs{T}' = \bs{Q}^T \bs{T} \bs{Q}$ where $\bs{Q}$ is orthogonal (rotation/reflection),
i.e. $\det(\bs{Q}) = \pm 1$. From the properties of the determinant we have:
\be
\det (\bs{T}') = \det(\bs{Q}^T \bs{T}\bs{Q}) = \det(\bs Q^T \bs T \bs Q) = (\det(\bs{Q}))^2 \det(\bs{T}) = \det(\bs{T}).
\ee

\newpage
\item  Let $\bs{F}$ be an arbitrary tensor.
\bi
\item Show that $\bs{F}\bs{F}^T$ and $\bs{F}^T\bs{F}$ are both symmetric tensors
\item  If $\bs{F}=\bs{QU}=\bs{VQ}$ where $\bs{Q}$ is orthogonal and $\bs{U,V}$ are symmetric, show that $\bs{U}^2=\bs{F}^T\bs{F}$ and $\bs{V}^2=\bs{F}\bs{F}^T$.
\ei

\underline{Solution:}
\bi
\item Using matrix notation:
\be 
\left(\bs{F}\bs{F}^T\right)^T = (\bs{F}^T)^T \bs{F}^T = \bs{F}\bs{F}^T
\ee
Similarly:
\be
\left(\bs{F}^T\bs{F}\right)^T = \bs{F}^T(\bs{F}^T)^T=\bs{F}^T\bs{F}
\ee

\item Using matrix notation and from orthogonality of $\bs{Q}$:
\be
\bs{F}^T\bs{F} = \bs{U}^T \bs{Q}^T\bs{Q} \bs{U} = \bs{U^T} \bs{U} = \bs{U}^2
\ee
and:
\be 
\bs{F}\bs{F}^T = \bs{V}\bs{Q}\bs{Q}^T\bs{V}^T = \bs{V}\bs{V}^T = \bs{V}^2
\ee
\ei


\item  Let $\{\bs{\hat{e}}_i\}_{i = 1}^3$  and $\{\bs{\hat{e}}'_i\}_{i = 1}^3$ be two 3-dimensional orthonormal
bases. If $\bs{\hat{e}}_i' = Q_{mi}\bs{\hat{e}}_m$,
\bi
\item  Verify $Q_{mi}Q_{mj}=Q_{im}Q_{jm}=\delta_{ij}$
\item  Show that $\bs{\hat{e}}_i=Q_{im}\bs{\hat{e}}'_m$
\ei

\underline{Solution:}

\bi
\item We have that $\bs{\hat{e}}_i \cdot \bs{\hat{e}}_j=\bs{\hat{e}}'_i \cdot \bs{\hat{e}}'_j=\delta_{ij}$. 
Hence:
\be
\delta_{ij}=\bs{\hat{e}}'_i \cdot \bs{\hat{e}}'_j=Q_{mi}\bs{\hat{e}}_m \cdot Q_{nj}\bs{\hat{e}}_n=Q_{mi}Q_{nj} \bs{\hat{e}}_m \cdot \bs{\hat{e}}_n =
Q_{mi}Q_{nj}\delta_{mn}=Q_{mi}Q_{mj}
\ee
Hence $\bs{Q}^T\bs{Q}=\bs{I}$, i.e. $\bs{Q}$ is orthogonal and also $\bs{Q}\bs{Q}^T=\bs{I}$ or $Q_{im}Q_{jm}=\delta_{ij}$.
Finally, by multiplying $\bs{\hat{e}}'_i=Q_{mi}\bs{\hat{e}}_m$ with $Q_{ni}$, we obtain:
\be
\begin{array}{l}
Q_{ni} \bs{\hat{e}}'_i=Q_{ni} Q_{mi}\bs{\hat{e}}_m \\
=\delta_{nm} \bs{\hat{e}}_m \\
=\bs{\hat{e}}_n, \\
\to \bs{\hat{e}}_i=Q_{im} \bs{\hat{e}}'_m.
\end{array}
\ee

\ei

\end{enumerate}



\end{document}
