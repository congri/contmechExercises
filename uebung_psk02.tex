\documentclass{article}
\usepackage{nips07submit_e,times}
%\documentstyle[nips07submit_09,times]{article}

\title{\underline{MSE - WS 2016/17}\\
Continuum Mechanics}


\author{
Instructor: P.S. Koutsourelakis  \\
\texttt{p.s.koutsourelakis@tum.de} \\
}

% The \author macro works with any number of authors. There are two commands
% used to separate the names and addresses of multiple authors: \And and \AND.
%
% Using \And between authors leaves it to \LaTeX{} to determine where to break
% the lines. Using \AND forces a linebreak at that point. So, if \LaTeX{}
% puts 3 of 4 authors names on the first line, and the last on the second
% line, try using \AND instead of \And before the third author name.


%
%%%%%%%%%%%%%%%%%%%%%%%%%%%%%%%%%%%%%%%%%%%%%%%%%%%%%%%
\usepackage{amsmath,amssymb}
\usepackage{hyperref} 

\usepackage[dvips]{psfrag,graphicx}


\linespread{1.6}
\newcommand{\ee}{\end{equation}}
\newcommand{\be}{\begin{equation}}
\newcommand{\ec}{\end{center}}
\newcommand{\bc}{\begin{center}}
\newcommand{\eea}{\end{eqnarray}}
\newcommand{\bea}{\begin{eqnarray}}
\newcommand{\bd}{\begin{description}}
\newcommand{\ed}{\end{description}}
\newcommand{\bi}{\begin{itemize}}
\newcommand{\ei}{\end{itemize}}
\newcommand{\pa}{\partial}
\newcommand{\bs}{\boldsymbol}
\def\RR{ \mathbb R}
\newcommand{\refeq}[1]{Equation (\ref{#1})}
%%%%%%%%%%%%%%%%%%%%%%%%%%%%%%%%%%%%%%%%%%%%%%%%%%%%%%
%

\begin{document}

\makeanontitle

%\begin{abstract}
%\input{abstract}
%\end{abstract}

\section*{\"Ubung - Week 02}


\begin{enumerate}
\item Let $\bs{A}$ be a matrix with columns consisting of the vectors $\bs a, \bs b, \bs c$. Show that:
\be
\det(\bs{A})=(\bs{a} \times \bs{b}) \cdot \bs{c}
\ee
\item In standard basis, a tensor $\bs{T}$ has the matrix
\be
\bs{T}=\left[\begin{array}{lll}
   1&  1 & 0 \\          1 & 1 & 0 \\ 0 & 0 & 2 
             \end{array}
\right].
\ee
\bi
\item  Find the principal values and three mutually perpendicular principal directions.
\item Find the maximum and minimum values that the diagonal entries of $\bs{T}$ can take under various coordinate
systems.
\ei

\item Let $r^2=x_k x_k$. Find $\frac{\pa r}{\pa x_j}$ and $\frac{\pa^2 r}{\pa x_i \pa x_j}$.

\item  Let $\bs{D}$ be a constant tensor whose components do not depend upon
the coordinates. Show that:
\be
\nabla (\bs{D} \bs{x})=\bs{D}
\ee

\item  Consider the scalar field $\phi=x_1^2+3x_1x_2+2x_3$
\bi
\item  Find the unit vector normal to the surface of constant $\phi$ at the origin and at $(1,0,1)$. 
\item What is the maximum value of the directional derivative of $\phi$ at the origin/ at (1,0,1)? 
\item  Evaluate $d\phi/d r$ at the origin if $d\bs{r}=dr(\bs{\hat{e}}_1+\bs{\hat{e}}_3)$
\ei

\item If $\phi(\bs{x})$ and $\psi(\bs{x})$ are scalar functions, show that for any domain $\mathcal{B}$ with bounding
surface $\pa \mathcal{B}$:
\be
\int_{\mathcal B}( \psi \phi_{,ii}+\psi_{,i}\phi_{,i})~dV=\int_{\pa \mathcal{B}} \psi \phi_i n_i dA
\ee
where $n_i$ is the unit outward normal, $dV$ implies integration over the volume and $dA$ integration over the bounding surface's area.

\item (Exam WS14) Consider  a three-dimensional material body occupying a domain $\mathcal{B}$ with a  a volume $V$. Let
$x_i$ denote the position vector,  $S$ denote the boundary surface and $n_i$ the unit,  outward normal vector. Explain why
\be
\int_S x_i n_j dS=\delta_{ij} V
\ee
where the first integral is over the bounding surface.

\item (Exam WS15) Consider a material body occupying the three-dimensional unit cube $[0, 1]^3$.
Let $x_i$ denote the position vector, $S$ denote the external boundary surface of the cube and $n_i$
the unit, outward normal vector to $S$. For the following integral over the boundary surface
$S$:
\be
\int_S x_i x_j n_j dS
\ee
\begin{itemize}
\item[a)] Which order tensor will the result be (e.g. $0^{th}, 1^{st}, 2^{nd}$ etc)?
\item[b)] Compute the components of this tensor
\end{itemize}





\end{enumerate}

\end{document}
