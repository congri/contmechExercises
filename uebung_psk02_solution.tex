\documentclass{article}
\usepackage{nips07submit_e,times}
%\documentstyle[nips07submit_09,times]{article}

\title{\underline{MSE - WS 2016/17}\\
Continuum Mechanics}


\author{
Instructor: P.S. Koutsourelakis  \\
\texttt{p.s.koutsourelakis@tum.de} \\
}

% The \author macro works with any number of authors. There are two commands
% used to separate the names and addresses of multiple authors: \And and \AND.
%
% Using \And between authors leaves it to \LaTeX{} to determine where to break
% the lines. Using \AND forces a linebreak at that point. So, if \LaTeX{}
% puts 3 of 4 authors names on the first line, and the last on the second
% line, try using \AND instead of \And before the third author name.


%
%%%%%%%%%%%%%%%%%%%%%%%%%%%%%%%%%%%%%%%%%%%%%%%%%%%%%%%
\usepackage{amsmath,amssymb}
\usepackage{hyperref} 
\usepackage{paralist} %enumerate with a) b) c) etc.

\usepackage[dvips]{psfrag,graphicx}


\linespread{1.6}
\newcommand{\ee}{\end{equation}}
\newcommand{\be}{\begin{equation}}
\newcommand{\ec}{\end{center}}
\newcommand{\bc}{\begin{center}}
\newcommand{\eea}{\end{eqnarray}}
\newcommand{\bea}{\begin{eqnarray}}
\newcommand{\bd}{\begin{description}}
\newcommand{\ed}{\end{description}}
\newcommand{\bi}{\begin{itemize}}
\newcommand{\ei}{\end{itemize}}
\newcommand{\pa}{\partial}
\newcommand{\bs}{\boldsymbol}
\def\RR{ \mathbb R}
\newcommand{\refeq}[1]{Equation (\ref{#1})}
%%%%%%%%%%%%%%%%%%%%%%%%%%%%%%%%%%%%%%%%%%%%%%%%%%%%%%
%

\begin{document}

\makeanontitle

%\begin{abstract}
%\input{abstract}
%\end{abstract}

\section*{\"Ubung - Week 02}


\begin{enumerate}
\item Let $\bs{A}$ be a matrix with columns consisting of the vectors $\bs a,\bs b, \bs c$. Show that:
\be
\det(\bs{A})=(\bs{a} \times \bs{b}) \cdot \bs{c}
\ee

\underline{Solution:} \\
The matrix $[\bs{A}]$ is given by:
\be
[\bs{A}]=\left[ \begin{array}{lll} a_1 & b_1 & c_1 \\ a_2 & b_2 & c_2\\ a_3 & b_3 & c_3 \end{array} \right],
\ee
where $a_i,b_i,c_i$ are the components of the vectors $\bs{a,b,c}$.
From definition, its determinant is:
\be
\begin{array}{ll}
\det [\bs{A}] &=a_1 \left| \begin{array}{ll}  b_2 & c_2 \\  b_3 & c_3 \end{array} \right|-a_2\left| \begin{array}{ll}  b_1 & c_1 \\  b_3 & c_3 \end{array}
\right|+a_3 \left|\begin{array}{ll}  b_1 & c_1 \\  b_2 & c_2 \end{array} \right| \\
&=a_1b_2c_3-a_1b_3c_2-a_2b_1c_3+a_2b_3c_1+a_3b_1c_2-a_3b_2c_1.
\end{array}
\ee
Furthermore, the product $(\bs{a} \times \bs{b}) \cdot \bs{c}$ can be written as:
\be
(\bs{a} \times \bs{b}) \cdot \bs{c}=\epsilon_{ijk}a_ib_jc_k
\ee
where $\epsilon_{ijk}$ is the permutation symbol in section 2.4. of Lai et al. Expanding the summations in the expression above, we get (see Eqs. 2.4.2 in Lai
et al.)
\be
(\bs{a} \times \bs{b}) \cdot \bs{c}=a_1b_2c_3-a_1b_3c_2-a_2b_1c_3+a_2b_3c_1+a_3b_1c_2-a_3b_2c_1=\det [\bs{A}].
\ee
\item In standard basis, a tensor $\bs{T}$ has the matrix
\be
\bs{T}=\left[\begin{array}{lll}
   1&  1 & 0 \\          1 & 1 & 0 \\ 0 & 0 & 2 
             \end{array}
\right]
\ee
\bi
\item Find the principal values and three mutually perpendicular principal directions.
\item Find the maximum and minimum values that the diagonal entries of $\bs{T}$ can take under various coordinate
systems.
\ei

\underline{Solution:} \\
\bi
\item The principal values (eigenvalues) $\lambda$ can be found by solving the following equation:
\be
det(\bs{T}-\lambda \bs{I})=0 \to \left| \begin{array}{lll}
   1-\lambda&  1 & 0 \\          1 & 1-\lambda & 0 \\ 0 & 0 & 2-\lambda 
             \end{array} \right|=0
             \ee
which           leads to:
\be
(2-\lambda)((1-\lambda)^2-1)=0
\ee
The three roots are $\lambda_1=0, \lambda_{2,3}=2$, i.e. the second and third eigenvalue twofold degenerate. The corresponding principal directions
$\bs{n}_1,\bs{n}_2,\bs{n}_3$ can be found by solving the following equations:
\be
(\bs{T}-\lambda \bs{I})\bs{n}=0.
\ee
We obtain:
\begin{align}
[\bs{n}_1]&=\frac{1}{\sqrt{2}}\left[ \begin{array}{l} 1 \\ -1 \\0 \end{array} \right] && \textit{eigenvector to the eigenvalue $\lambda_1 = 0$.}
\end{align}
The eigenspace corresponding to the degenerate eigenvalue $\lambda_{2,3} = 2$ is two-dimensional and can be spanned by the vectors
\begin{align}
[\bs{n}_2]&= \left[ \begin{array}{l} 0 \\ 0 \\ 1 \end{array} \right], \quad [\bs{n}_3]=\frac{1}{\sqrt{2}} \left[ \begin{array}{l} 1 \\
1 \\ 0 \end{array} \right] && \textit{to the eigenvalue $\lambda_{2,3}$.}
\end{align}
This is an arbitrary choice. Keep in mind that any linear combination of the vectors $\bs n_2, \bs n_3$ is also an eigenvector of $\bs T$ to the eigenvalue 2.
\item The largest and smallest principal value correspond to the maximum and minimum of the diagonal entries of the tensor in any coordinate system (see
section 2.24 of Lai et al.). In this problem these are $2$ and $0$, respectively.
\ei 

\item Let $r^2 = x_k x_k$. Find $\frac{\pa r}{\pa x_j}$ and $\frac{\pa^2 r}{\pa x_i \pa x_j}$.

\underline{Solution:}

We call
\begin{align}
s(\bs x) &= x_k x_k, \\
r(s(\bs x)) &= \sqrt{x_k x_k} = \sqrt{s(\bs x)}.
\end{align}
Using the chain rule,
\begin{align}
\frac{\pa r(s(\bs x))}{\pa x_j} &= \frac{\pa r}{\pa s} \frac{\pa s}{\pa x_j}, \\
\frac{\pa r}{\pa s} &= \frac{1}{2\sqrt{s}}, \\
\frac{\pa s}{\pa x_j} &= \delta_{jk} x_k + x_k \delta_{jk} = 2x_j.
\end{align}
Thus
\be
\frac{\pa r(s(\bs x))}{\pa x_j} = \frac{\pa r}{\pa s} \frac{\pa s}{\pa x_j} = \frac{x_j}{\sqrt{x_k x_k}}.
\ee

\begin{align}
\frac{\pa^2 r}{\pa x_i \pa x_j} &= \frac{\pa }{\pa x_i}\left(\frac{\pa r}{\pa x_j}\right) && \\
&= \frac{\pa x_j}{\pa x_i} \frac{1}{\sqrt{x_k x_k}} + x_j \frac{\pa}{\pa x_i} \left(\frac{1}{\sqrt{x_k x_k}}\right) &&\textit{product rule} \\
&= \frac{\delta_{ij}}{\sqrt{x_k x_k}} + x_j \left(-\frac{1}{2}\right) \left(x_k x_k\right)^{-3/2} 2x_i &&\textit{chain rule in 2. term} \\
&= \frac{\delta_{ij}}{\sqrt{x_k x_k}} - \frac{x_i x_j}{\left(x_k x_k\right)^{3/2}}.
\end{align}

\item Let $\bs{D}$ be a constant tensor whose components do not depend upon
the coordinates. Show that:
\be
\nabla (\bs{D} \bs{x})=\bs{D}.
\ee

\underline{Solution:} \\
Let  $\bs{v}=\bs{D} \bs{x}$. Since $\bs{v}$ is a vector, its gradient will be a tensor. We use indicial notation:
\be
\begin{array}{ll}
(\nabla \bs{v})_{ij}& =\frac{\pa v_i}{\pa x_j}=\frac{\pa  }{\pa x_j}(D_{ik}x_k) \\
& = \frac{\pa D_{ik}}{\pa x_j} x_k +D_{ik} \frac{\pa  x_k}{\pa x_j} \\
& = 0 x_k +D_{ik} \delta_{kj}=D_{ij}
\end{array}
\ee

\item Consider the scalar field $\phi=x_1^2+3x_1x_2+2x_3$
\bi
\item Find the unit vector normal to the surface of constant $\phi$ at the origin and at $(1,0,1)$. 
\item What is the maximum value of the directional derivative of $\phi$ at the origin at (1,0,1)? 
\item Evaluate $d\phi/d r$ at the origin if $d\bs{r}=dr(\bs{\hat{e}}_1+\bs{\hat{e}}_3)$
\ei

\underline{Solution:} \\
\bi
\item The unit vector normal to the surface of constant $\phi$ at the origin and at $(1,0,1)$ is obtained from the {\em gradient } of $\phi$ at this point (see
section 2.27 in Lai et al.). In particular:
\be
[\nabla \phi]=\left[ \begin{array}{c} 2x_1+3x_2 \\ 3x_1 \\2\end{array} \right]
\ee
which at the origin becomes $\left[ \begin{array}{l} 0 \\ 0 \\2 \end{array} \right]$ and at $(1,0,1)$,  $\left[ \begin{array}{l} 2 \\ 3 \\2 \end{array}
\right]$. % with length $\sqrt{2^2+3^2+2^2}=\sqrt{17}$.
Hence the unit normal vectors at these two points are:
\be
  \left[ \begin{array}{l} 0 \\ 0 \\1 \end{array} \right]  \textrm{  and   }   \frac{1}{\sqrt{17}}\left[ \begin{array}{l} 2 \\ 3 \\2 \end{array} \right],
\ee
respectively. 
\item The maximum value of the directional derivative of $\frac{\pa \phi}{\pa dr}$ at any point is given by $|\nabla \phi|$ (see section 2.27 in Lai et al.)
which are equal to $2$ and $\sqrt{17}$ at the origin and $(1,0,1)$, respectively.

\item From the definition of the gradient:
\be
d \phi= \nabla \phi \cdot d\bs{r}
\ee
At the origin $\nabla \phi=2 \bs{\hat{e}}_3$ and for $d\bs{r}=dr(\bs{\hat{e}}_1+\bs{\hat{e}}_3)$ we obtain:
\be
d \phi=2 \bs{\hat{e}}_3 \cdot dr(\bs{\hat{e}}_1+\bs{\hat{e}}_3) \to  \frac{d \phi}{ dr}=2.
\ee

\ei

\item If $\phi(\bs{x})$ and $\psi(\bs{x})$ are scalar functions, show that for any domain $\mathcal{B}$ with bounding surface $\pa \mathcal{B}$:
\be
\int_{\mathcal B}(\psi \phi_{,ii}+\psi_{,i}\phi_{,i})~dV=\int_{\pa \mathcal{B}} \psi \phi_{,i} n_i dA,
\ee
where $n_i$ is the unit outward normal.


\underline{Solution:} \\
We observe that:
\be
\psi \phi_{,ii}+\psi_{,i}\phi_{,i}=(\psi \phi_{,i})_{,i}.
\ee
Hence by application of the divergence theorem:
\be
\int( \psi \phi_{,ii}+\psi_{,i}\phi_{,i})~dV=\int(\phi \phi_{,i})_{,i} ~dV= \int_{\pa \mathcal{B}} \psi \phi_{,i} n_i dA,
\ee
where $n_i$ is the unit outward normal.

\item (Exam WS14) Consider a three-dimensional material body occupying a domain $\mathcal{B}$ with a  a volume $V$. Let  $x_i$ denote the position vector,  $S$ denote the boundary surface and $n_i$ the unit,  outward
normal vector. Explain why
\be
\int_S x_i n_j dS=\delta_{ij} V
\ee
where the first integral is over the bounding surface.

\underline{Solution:}
\begin{align}
\int_S x_i n_j dS &= \int_S x_i \bs e_j \cdot \bs n dS = \int_V \textrm{div}(x_i \bs e_j) dV = \delta_{ij} \int_V dV = \delta_{ij} V.
\end{align}

\item (Exam WS15) Consider a material body occupying the three-dimensional unit cube $[0, 1]^3$.
Let $x_i$ denote the position vector, $S$ denote the external boundary surface of the cube and $n_i$
the unit, outward normal vector to $S$. For the following integral over the boundary surface
$S$:
\be
\int_S x_i x_j n_j dS
\ee
you are asked to find:
\begin{compactenum}[a)]
\item Which order tensor will the result be (e.g. $0^{th}, 1^{st}, 2^{nd}$ etc)?
\item Compute the components of this tensor
\end{compactenum}

\underline{Solution:}

\begin{compactenum}[a)]
\item $i$ is a free index and $j$ is a dummy index, hence the result will be a first order tensor/vector.
\item
\begin{align}
J_i &:= \int_S x_i x_j n_j dS && \\
&= \int_S x_i \bs x \cdot \bs n dS && \\
&= \int_V \textrm{div}(x_i \bs x) dV && \textit{divergence theorem} \\
&= \int_V (x_i x_j)_{,j} dV = \int_V (\delta_{ij} x_j + 3x_i)dV  && \\
&= 4 \int_V x_i dV && \\
&= 2.
\end{align}
Hence
\be
\bs J =
\begin{pmatrix}
2\\
2\\
2
\end{pmatrix}.
\ee
\end{compactenum}


\end{enumerate}



\end{document}
