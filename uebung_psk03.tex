\documentclass{article}
\usepackage{nips07submit_e,times}
\usepackage{color}
%\documentstyle[nips07submit_09,times]{article}

\title{\underline{MSE - WS 2016/17}\\
Continuum Mechanics}


\author{
Instructor: P.S. Koutsourelakis  \\
\texttt{p.s.koutsourelakis@tum.de} \\
}

% The \author macro works with any number of authors. There are two commands
% used to separate the names and addresses of multiple authors: \And and \AND.
%
% Using \And between authors leaves it to \LaTeX{} to determine where to break
% the lines. Using \AND forces a linebreak at that point. So, if \LaTeX{}
% puts 3 of 4 authors names on the first line, and the last on the second
% line, try using \AND instead of \And before the third author name.


%
%%%%%%%%%%%%%%%%%%%%%%%%%%%%%%%%%%%%%%%%%%%%%%%%%%%%%%%
\usepackage{amsmath,amssymb}
\usepackage{hyperref} 

\usepackage[dvips]{psfrag,graphicx}


\linespread{1.6}
\newcommand{\ee}{\end{equation}}
\newcommand{\be}{\begin{equation}}
\newcommand{\ec}{\end{center}}
\newcommand{\bc}{\begin{center}}
\newcommand{\eea}{\end{eqnarray}}
\newcommand{\bea}{\begin{eqnarray}}
\newcommand{\bd}{\begin{description}}
\newcommand{\ed}{\end{description}}
\newcommand{\bi}{\begin{itemize}}
\newcommand{\ei}{\end{itemize}}
\newcommand{\pa}{\partial}
\newcommand{\bs}{\boldsymbol}
\newcommand{\tbf}{\textbf}
\newcommand{\tm}{\textrm}
\def\RR{ \mathbb R}
\newcommand{\refeq}[1]{Equation (\ref{#1})}
\newcommand{\ul}{\underline}
\newcommand{\emat}{\end{pmatrix}}
\newcommand{\bmat}{\begin{pmatrix}}
\newcommand{\esmat}{\end{smallmatrix}\right)}
\newcommand{\bsmat}{\left(\begin{smallmatrix}}
%%%%%%%%%%%%%%%%%%%%%%%%%%%%%%%%%%%%%%%%%%%%%%%%%%%%%%

%Exercise or solution?
\newif\ifsolution
\solutiontrue			%uncomment to show solution



\begin{document}

\makeanontitle

\section*{\"Ubung - Week 03}
\begin{enumerate}
\item Consider the motion
\be
x_1 = at+X_1, \quad x_2=X_2, \quad x_3= X_3,
\ee 
where the material coordinates $X_I$ designate the
position of a particle at $t=0$.
\bi
\item Determine the velocity and acceleration of a particle in both material and spatial description.

\ifsolution
\underline{Solution:}

In material description the velocity vector is given by:
\be
\bs{V}(\bs{X},t)=\frac{\pa \bs{x}}{\pa  t} \to [\bs{V}(\bs{X},t)]=\left[\begin{array}{l}  a \\ 0 \\ 0 \end{array} \right].
\ee
The acceleration vector is given by:
\be
\bs{A}(\bs{X},t)=\frac{\pa \bs{V}}{\pa  t} \to [\bs{A}(\bs{X},t)]=\left[\begin{array}{l}  0 \\ 0 \\ 0 \end{array} \right].
\ee

To obtain the velocity and acceleration fields in  spatial description one must invert the deformation map
$\bs{x}=\bs{\phi}(\bs{X},t)$, i.e. find $\bs{X}=\bs{\phi}^{-1}(\bs{x},t)$ and substitute in the velocity and
acceleration expressions in the material description. In this problem one can easily find that:
\be
X_1=x_1-at, \quad X_2=x_2, \quad x_3=X_3.
\ee
Hence the velocity vector in spatial description is given by:
\be
\bs{v}(\bs{x},t)=\bs{V}(\bs{\phi}^{-1}(\bs{x}, t), t) \rightarrow [\bs{v}(\bs{x},t)]=\left[\begin{array}{l}  a \\ 0 \\ 0
\end{array} \right].
\ee
Similarly for the acceleration vector:
\be
\bs{a}(\bs{x},t)=\bs{A}(\bs{\phi}^{-1}(\bs{x},t), t) \rightarrow [\bs{a}(\bs{x},t)]=\left[\begin{array}{l}  0 \\ 0 \\ 0
\end{array} \right].
\ee
\fi

\item  If the temperature field in spatial description is given by $\theta=Ax_1$, what is its material description?
Find the material derivative of $\theta$ using both descriptions of the temperature field.

\ifsolution
\underline{Solution:}

Using the the deformation map from above we obtain that the temperature field in material description is:

\be
\theta(\bs{X}) = \theta(at + X_1)=A(at + X_1).
\ee


\textbf{Note that the expression above involves a slight abuse of notation that is nevertheless frequently encountered
in continuum-mechanics literature. We used $\theta$ to denote the temperature functions in both material and spatial
descriptions. In actuality, they are different functions with different arguments}.


The material derivative in material description is:
\be
\frac{D \theta}{D t}=\frac{\pa \theta(\bs{X},t)}{\pa t}=Aa
\ee
The material derivative when $\theta$ is given in spatial description is
\be
\begin{array}{ll}
\frac{D \theta}{D t}&=\frac{\pa \theta(\bs{x},t)}{\pa t}+\frac{\pa \theta(\bs{x},t)}{\pa  x_i} \cdot \frac{\pa x_i}{\pa t} \\
& = \frac{\pa \theta(\bs{x},t)}{\pa t}+\frac{\pa \theta(\bs{x},t)}{\pa \bs{x}} \cdot v_i(\bs{x},t).
\end{array}
\ee
where $\bs{v}(\bs{x},t)$ is the velocity (in spatial description) given above.
Hence:
\be
\frac{D \theta}{D t} = 0 + \left[\begin{array}{lll}  A & 0 & 0 \end{array} \right]^T \left[\begin{array}{l}  a \\ 0 \\ 0
\end{array} \right] = Aa.
\ee
which coincides with the result found previously using the material description.
\fi

\item Repeat the previous calculations if the temperature field is $\theta=Bx_2$.

\ifsolution
\underline{Solution:}

Similarly:
\be
\theta(\bs{X})=\theta(X_2)=BX_2.
\ee
The material derivative in spatial description is:
\be
\frac{D \theta}{D t}=\frac{\pa \theta(\bs{X},t)}{\pa t} = 0
\ee
The material derivative when $\theta$ is given in spatial description is:
\be
\begin{array}{ll}
\frac{D \theta}{D t}&=\frac{\pa \theta(\bs{x},t)}{\pa t}+\frac{\pa \theta(\bs{x},t)}{\pa x_i} \cdot \frac{\pa x_i}{\pa t} \\
& = \frac{\pa \theta(\bs{x},t)}{\pa t}+\frac{\pa \theta(\bs{x},t)}{\pa x_i} \cdot v_i(\bs{x},t) \\
& = 0 +\left[\begin{array}{lll}  0 , B ,0 \end{array} \right] \left[\begin{array}{l}  a \\ 0 \\ 0 \end{array} \right] =
0
\end{array}
\ee
\fi
\ei

\item Consider the motion 
\be
x_1 = bX_2 t^2 + X_1, \quad x_2 = kX_2 t + X_2, \quad  x_3 = X_3.
\ee
\bi
\item At $t = 0$, the corners of a unit square are at $A(0,0,0)$, $B(0,1,0)$, $C(1,1,0)$, $D(1,0,0)$.  Sketch the
deformed shape of the square at $t = 2$.

\ifsolution
\underline{Solution:}

At $t=2$ the corners of the square will be located at:
\bi
\item $A \to A'$ with coordinates:
\be
x_1= b ~0 ~2^2+0=0, \quad  x_2=k~0 ~2+0=0, \quad   x_3 = 0
\ee


\item $B \to B'$ with coordinates:
\be
x_1=b~1~ 2^2+0=4b,\quad  x_2=k~1~2 +1=2k+1, \quad  x_3 = 0
\ee


\item $C \to C'$ with coordinates:
\be
x_1=b~1~ 2^2+1=4b+1,\quad  x_2=k~1~2 +1=2k+1, \quad  x_3 = 0
\ee


\item $D \to D'$ with coordinates:
\be
x_1=b~0~ 2^2+1=1,\quad  x_2=k~0~2 +0=0, \quad  x_3 = 0
\ee
\ei
\fi
\item Obtain the spatial description of the velocity field.

\ifsolution
\underline{Solution:}

In {\em material description} the velocity field is:
\be
\bs{V}(\bs{X},t)=\frac{\pa \bs{x}}{\pa  t} \rightarrow [\bs{V}(\bs{X},t)]=\left[\begin{array}{l}  2btX_2 \\ kX_2 \\ 0
\end{array} \right]
\ee
Furthermore, by inverting the deformation map, we obtain:
\be
X_1=x_1-bt^2 x_2/(1+kt), \quad X_2=x_2/(1+kt), \quad X_3=x_3.
\ee
Hence the velocity vector in spatial description is given by:
\be
\bs{v}(\bs{x},t)=\bs{V}(\bs{\phi}^{-1}(\bs{x},t),t) \rightarrow [\bs{v}(\bs{x},t)]=\left[\begin{array}{l}  2bt
\frac{x_2}{1+kt} \\ k\frac{x_2}{1+kt} \\ 0 \end{array} \right]
\ee
\textbf{Note that even though time does not appear \underline{explicitly} in the material description  it does so in
the spatial. This is because the former gives the velocity of a specific particle (initially found at $\bs{X}$) whereas
the latter gives the velocity at a specific spatial location $\bs{x}$ which is occupied by different particles at
different times.}
\fi

\item Obtain the spatial description of the acceleration field.

\ifsolution
\underline{Solution:}

In {\em material description} the acceleration field is:
\be
\bs{A}(\bs{X},t)=\frac{\pa \bs{V}}{\pa  t} \to [\bs{A}(\bs{X},t)]=\left[\begin{array}{l}  2bX_2 \\ 0 \\ 0 \end{array}
\right]
\ee

Hence the acceleration vector in spatial description is:
\be
\bs{a}(\bs{x},t)=\bs{A}(\bs{\phi}^{-1}(\bs{x},t),t) \to [\bs{a}(\bs{x},t)]=\left[\begin{array}{l}   2b \frac{x_2}{1+kt}
\\ 0 \\ 0 \end{array} \right]
\ee

\textbf{ Note that even though time does not appear \underline{explicitly} in the material description description it
does so in the spatial. This is because the former gives the acceleration of a specific  particle (initially found at
$\bs{X}$) whereas the latter gives the acceleration at a specific spatial location $\bs{x}$ which is occupied by
different particles at different times.}
\fi
\ei

\item  Consider the motion 
\be
x_1=k(s+X_1)t+X_1, \quad x_2=X_2, \quad x_3=X_3.
\ee
\bi
\item Find the velocity and acceleration as a function of time of a particle that is initially at the origin.

\ifsolution
\underline{Solution:}

A particle that is originally at the origin implies that it has material coordinates\newline $\bs{X}=(0,0,0)$.
Since we need to find velocity and acceleration of a specific particle, we must consider the {\em material description}.

In material description the velocity vector is given by:
\be
\bs{V}(\bs{X},t)=\frac{\pa \bs{x}}{\pa  t} \to [\bs{V}(\bs{X},t)]=\left[\begin{array}{l}  k(s+X_1) \\ 0 \\ 0 \end{array} \right]
\ee
Hence:
\be
[\bs{V}((0,0,0),t)]=\left[\begin{array}{l}  ks \\ 0 \\ 0 \end{array} \right]
\ee
i.e. the particle moves at constant velocity.

The acceleration vector is given by:
\be
\bs{A}(\bs{X},t)=\frac{\pa \bs{V}}{\pa  t} \to [\bs{A}(\bs{X},t)]=\left[\begin{array}{l}  0 \\ 0 \\ 0 \end{array} \right]
\ee
and:
\be
[\bs{A}((0,0,0),t)]=\left[\begin{array}{l}  0 \\ 0 \\ 0 \end{array} \right]
\ee
\fi
\item Find the velocity and acceleration as a function of time of the particles that are passing through the origin.

\ifsolution
\underline{Solution:}

Since we need to find velocity and acceleration at a specific location (occupied by different particles), we must
consider the {\em spatial description}.

To obtain the velocity and acceleration fields in  spatial description one must invert the deformation map
$\bs{x}=\bs{\phi}(\bs{X},t)$ i.e. find $\bs{X}=\bs{\phi}^{-1}(\bs{x},t)$ and substitute in the velocity and
acceleration expressions in the material description. In this problem one can easily find that:
\be
X_1 = \frac{x_1-kst}{1+kt}, \quad X_2 = x_2, \quad x_3 = X_3
\ee
Hence the velocity vector in spatial description is given by:
\be
\bs{v}(\bs{x},t)=\bs{V}(\bs{\phi}^{-1}(\bs{x},t),t) \to [\bs{v}(\bs{x},t)]=\left[\begin{array}{l} 
\frac{k(s+x_1)}{1+kt} \\ 0 \\ 0 \end{array} \right]
\ee
At the origin, i.e. for $\bs{x}=(0,0,0)$:
\be
 [\bs{v}((0,0,0),t)]=\left[\begin{array}{l}  \frac{ks}{1+kt} \\ 0 \\ 0 \end{array} \right]
\ee

Similarly for the acceleration vector:
\be
\bs{a}(\bs{x},t)=\bs{A}(\bs{\phi}^{-1}(\bs{x},t),t) \to [\bs{a}(\bs{x},t)]=\left[\begin{array}{l}  0 \\ 0 \\ 0 \end{array} \right]
\ee
and:
\be
[\bs{a}((0,0,0),t)]=\left[\begin{array}{l}  0 \\ 0 \\ 0 \end{array} \right]
\ee
\fi
\ei







\item Show that the velocity field $v_i = kx_i/(1+kt)$ corresponds to the motion $x_i = X_i(1+kt)$  and find the
acceleration of this motion in material description.

\ifsolution
\underline{Solution:}

If $x_i= X_i(1+kt) \leftrightarrow X_i=x_i/(1+kt)$, the velocity field in material description is given by:
\be
V_i(\bs{X},t)=\frac{\pa x_i}{\pa {t}} =kX_i
\ee
and the velocity field in spatial description will be:
\be
v_i=k \underbrace{\frac{x_i}{1+kt}}_{X_i}
\ee
\fi






\item A bar of length $L_0$ lies originally along the $X_1 \equiv x_1$ axis as seen in Figure \ref{fig:bar_rot}. The
 bar is \textbf{rigidly} rotated, counter-clockwise by $90^o$ degrees around point $O$ and it ends up along the $X_2
 \equiv x_2$  axis while retaining the same length $L_0$.
\begin{figure}[!h]
\centering
\psfrag{l0}{$L_0$}
\psfrag{x1}{$X_1 \equiv x_1$}
\psfrag{x2}{$X_2 \equiv x_2$}
\psfrag{O}{$O$}
\psfrag{rigid}{rotation $90^o$}
 \includegraphics[height=5cm]{FIGURES/bar_rotation.eps}
 \caption{Problem configuration}
 \label{fig:bar_rot}
\end{figure}
 \bi
 \item[a)] Find the deformation map for this motion.

\ifsolution 
\underline{\textbf{Solution:}}
 
 There are multiple ways to find the deformation map. We switch to polar coordinates:
 \be
 \bs X = \begin{pmatrix}
 r \cos \varphi \\
 r \sin \varphi
 \end{pmatrix}.
 \ee
 For a rotation of $90^\circ$, we simply add $90^\circ$ to the polar angle $\varphi$ to get
 \begin{align}
 \bs x &= \bs \phi(\bs X) = \begin{pmatrix}
 r\cos (\varphi + 90^\circ) \\
 r\sin (\varphi + 90^\circ)
 \end{pmatrix}
 = \begin{pmatrix}
 -r \sin \varphi \\
 r \cos \varphi
 \end{pmatrix}
 = \begin{pmatrix}
 -X_2 \\
 X_1
 \end{pmatrix}
 \end{align}
\fi

\item[b)] Find the displacement field in Lagrangian and Eulerian coordinates

\ifsolution
\underline{\textbf{Solution:}}

\begin{align}
\bs U(\bs X) &= \bs \phi(\bs X) - \bs X \\ 
&= \begin{pmatrix}
-X_1 - X_2 \\
X_1 - X_2
\end{pmatrix} \\
\bs u(\bs x) &= \bs x - \bs \phi^{-1}(\bs x) \\
&= \begin{pmatrix}
x_1 - x_2 \\
x_1 + x_2
\end{pmatrix}
\end{align}
\fi

 \item[c)] Compute the Lagrangian strain tensor

\ifsolution
\underline{\textbf{Solution:}}
 
The Lagrangian strain tensor is given by
 \begin{align}
 \bs E &= \frac{1}{2}\left(\nabla_{\bs X} \bs U + (\nabla_{\bs X} \bs U)^T + (\nabla_{\bs X} \bs U)^T\nabla_{\bs X} \bs
 U \right)
 \end{align}
 With
 \begin{align}
 \nabla_{\bs X} U &= \begin{pmatrix}
 -1 & -1 \\
 1 & -1
 \end{pmatrix}
 \end{align}
 we get
 \begin{align}
 \bs E &= \begin{pmatrix}
 0 & 0 \\
 0 & 0
 \end{pmatrix}
 \end{align}
 \underline{\textbf{Or:}}
 \begin{align}
 \bs E &= \frac{1}{2}(\bs F^T \bs F - \bs I)\\
 \bs F &= \nabla_{\bs X} \bs x = \begin{pmatrix}
 0 & -1 \\
 1 & 0
 \end{pmatrix}\\
 \bs F^T \bs F &= \begin{pmatrix}
 1 & 0 \\
 0 & 1
 \end{pmatrix}\\
 \bs E &= \begin{pmatrix}
 0 & 0 \\
 0 & 0
 \end{pmatrix}
 \end{align}
 \underline{\textbf{Or:}}
 
 Answer: This is a rigid body motion, thus there is no strain.
 \begin{align}
 \bs E &= \bs 0
 \end{align}
 \fi
\ei 







\item Given the following deformation in rectangular Cartesian coordinates: 
\be
x_1 = 3X_3, \quad x_2 = -X_1, \quad x_3 = -2 X_2,
\ee
determine (a) the deformation gradient $\bs{F}$, (b) the right Cauchy-Green tensor $\bs{C}$ and the right stretch tensor
$\bs{U}$, (c) the left Cauchy-Green tensor $\bs{b}$, (d) the rotation tensor $\bs{R}$ in the polar decomposition of
$\bs{F}$, (e) the Lagrangean finite strain tensor $\bs{E}$ , (f) the Eulerian finite  strain tensor $\bs{e}$ , (g) the
ratio of deformed volume to initial volume, and (h) the deformed area (magnitude and its normal) for the area whose
normal was in the direction of $\bs{\hat{e}}_2$ and whose magnitude was unity for the undeformed area.

\ifsolution
\underline{Solution:}

\bi
\item[a)] The deformation gradient tensor $\bs{F}$ has components:
\be
F_{iI}=\frac{\pa x_i }{\pa X_I} \to [\bs{F}]=\left[\begin{array}{lll} 0 & 0 & 3 \\-1 & 0 & 0 \\0 & -2 & 0 \end{array}
\right]
\ee

Note that the deformation gradient \underline{is not in general a symmetric tensor}.

\item[b)] the right Cauchy-Green tensor $\bs{C}$ is given by:
\be
\bs{C}=\bs{F}^T\bs{F} \to [\bs{C}]=\left[ \begin{array}{lll} 1 & 0 & 0 \\0 & 4 & 0 \\0 & 0 & 9 \end{array} \right]
\ee
Note that by definition the right Cauchy-Green tensor $\bs{C}$ is \underline{always} symmetric.

The right stretch tensor $\bs{U}$ arises from the polar decomposition of $\bs{F}$ i.e.:
\be
\bs{F}=\bs{R U} \to \bs{F}^T\bs{F}=\bs{U}^T \bs{R}^T \bs{R U} =\bs{U}^2
\ee
since $\bs{R}$ is orthogonal and $\bs{U}$ symmetric. From above we have that:
\be
\bs{U}^2=\bs{F}^T\bs{F}=\bs{C} \to [\bs{U}]=\left[\begin{array}{lll} 1 & 0 & 0 \\0 & 2 & 0 \\0 & 0 & 3 \end{array}
\right]
\ee
(Note: Do you remember how to find the square root of $\bs{C}$ if it is not diagonal?)
\item[c)] the left Cauchy-Green tensor $\bs{b}$ is given by:
\be
\bs{b}=\bs{F}\bs{F}^T \to [\bs{b}]=\left[ \begin{array}{lll} 9 & 0 & 0 \\0 & 1 & 0 \\0 & 0 & 4 \end{array} \right]
\ee
Note that by definition the left Cauchy-Green tensor $\bs{b}$ is \underline{always} symmetric.
\item[d)] From $\bs{F}=\bs{R U}$ we obtain that:
\be
\bs{R}=\bs{F}\bs{U}^{-1} \to [\bs{R}]=\left[ \begin{array}{lll} 0 & 0 & 1 \\-1 & 0 & 0 \\0 & -1 & 0 \end{array} \right]
\ee
(Note: can you verify that $\bs{R}$ is orthogonal?)
\item[e)] the Lagrangean finite  strain tensor $\bs{E}$  is given by:
\be
\bs{E}=\frac{1}{2}(\bs{C} -\bs{I}) \to [\bs{E}]=\left[ \begin{array}{lll} 0 & 0 & 0 \\0 & 3/2 & 0 \\0 & 0 & 4
\end{array} \right]
\ee
\item[f)] the Eulerian finite  strain tensor $\bs{e}$ is given by:
\be
\bs{e}=\frac{1}{2}(\bs{I} -\bs{b}^{-1}) \to [\bs{e}]=\left[ \begin{array}{lll} 4/9 & 0 & 0 \\0 & 0 & 0 \\0 & 0 & 3/8
\end{array} \right]
\ee
\item[g)] the ratio of deformed volume $dv$ to initial volume $dV$ is given by (Lai et al. section 3.28):
\be
\frac{dv}{dV} = \det(\bs{F}) = 6
\ee
\item[h)] The change in area is given by (Lai et al., section 3.27):
\be
d\bs{a}=\det(\bs{F})\bs{F}^{-T} d\bs{A}
\ee
where $d\bs{A}=dA \bs{N}$ is a \underline{vector} perpendicular to the undeformed area, $dA$ is the undeformed area
(scalar) and $\bs{N}$ is the unit normal to the undeformed area. Similarly, $d\bs{a}=da \bs{n}$ is a \underline{vector}
perpendicular to the deformed area, $da$ is the deformed area (scalar) and $\bs{n}$ is the unit normal to the deformed
area. In our case:
\be
d\bs{A}=1~\bs{\hat{e}}_2 \to [d\bs{A}] = \left[\begin{array}{l} 0 \\ 1 \\ 0 \end{array}\right]
\ee
and since $[\bs{F}^{-T}]=[(\bs{F}^{-1})^T]=\frac{1}{6}\left[ \begin{array}{lll} 0 & -6 & 0 \\0 & 0 & -3 \\2 & 0 & 0
\end{array} \right]^T=\frac{1}{6}\left[ \begin{array}{lll} 0 & 0 & 2 \\-6 & 0 & 0\\ 0 & -3 & 0 \end{array} \right]$:
\be
[d\bs{a}]=6~\frac{1}{6}\left[\begin{array}{lll} 0 & 0 & 2 \\-6 & 0 & 0\\ 0 & -3 & 0 \end{array} \right]
\left[\begin{array}{l} 0 \\ 1 \\ 0 \end{array}\right]=\left[\begin{array}{l} 0 \\ 0 \\ -3 \end{array}\right]
\ee
i.e. the area will increase by a factor of $3$ and the unit normal in the deformed state will be in the direction of
$-\bs{\hat{e}}_3$
\ei
\fi

\item  Given the following right Cauchy-Green deformation tensor at a point:
\be
[\bs{C}]=\left[ \begin{array}{ccc} 9 &0 &0 \\ 0 & 4 & 0 \\ 0 & 0 & 0.36 \end{array} \right]
\ee
\bi
\item  Find the stretch for the material elements that were in the direction of $\bs{\hat{e}}_1$, $\bs{\hat{e}}_2$ and
$\bs{\hat{e}}_3$.

\ifsolution
\underline{Solution:}

We have established that an infinitesimal fiber $d\bs{X}$ deforms to:
\be
d\bs{x}=\bs{F}d\bs{X} \quad \tm{or} \quad dx_i=F_{iI}dX_I
\ee
and the relationship between undeformed $dX=|d\bs{X}|$ and deformed lengths $dx=|d\bs{x}|$ is:
\be
(dx)^2=d\bs{x}_i d\bs{x}_i=F_{iI}dX_I F_{iJ}dX_J=dX_I (F_{iI} F_{iJ}) dX_J=dX_I C_{IJ} dX_J
\ee
where $\bs{C}$ is the right Cauchy-Green deformation tensor. Hence for elements in the direction $\bs{\hat{e}}_1$:
\be
 (dx)^2 = C_{11}(dX)^2 \to \textrm{ stretch ratio is:} \quad \left|\frac{dx}{dX}\right| = \sqrt{C_{11}} = 3
\ee
Similarly for $\bs{\hat{e}}_2$:
\be
 (dx)^2 = C_{22}(dX)^2 \to \textrm{ stretch ratio is:} \quad \left|\frac{dx}{dX}\right| = \sqrt{C_{22}} = 2
\ee
and for $\bs{\hat{e}}_3$:
\be
 (dx)^2 = C_{33}(dX)^2 \to \textrm{ stretch ratio is:} \quad \left|\frac{dx}{dX}\right| = \sqrt{C_{33}} = 0.6
\ee
\fi
\item  Find the stretch for the material element that was in the direction of $\bs{\hat{e}}_1+\bs{\hat{e}}_2$.

\ifsolution
\underline{Solution:}

Consider an element $d\bs{X}$ with undeformed length $d{X}$ in the direction of $\bs{\hat{e}}_1+\bs{\hat{e}}_2$, i.e.
$d\bs{X}=d{X}\frac{1}{\sqrt{2}} (\bs{\hat{e}}_1+\bs{\hat{e}}_2)$ or equivalently:
\be
[d\bs{X}] = \frac{dX}{\sqrt{2}} \bmat 1 \\1 \\ 0 \emat
\ee
Then from the relation above the deformed length $dx$ is given by:
\begin{align}
(dx)^2 &= d\bs{X}^T \bs{C} d\bs{X} = \frac{dX}{\sqrt{2}} \bmat 1 &1 & 0\emat \bmat 9 &0 &0 \\ 0 & 4 & 0 \\ 0 & 0 & 0.36
\emat \frac{dX}{\sqrt{2}} \bmat 1 \\1 \\ 0 \emat \\ \nonumber
&= \frac{13}{2} (dX)^2
\end{align}
and the stretch ratio is:
\be
\left|\frac{dx}{dX}\right| = \sqrt{\frac{13}{2}}.
\ee
\fi

\item  Find $\cos \theta$, where $\theta$ is the angle between the deformed elements $d\bs{x}^{(1)}$ and
$d\bs{x}^{(2)}$ corresponding to  $d\bs{X}^{(1)}=dS_1 \bs{\hat{e}}_1$ and $d\bs{X}^{(2)}=dS_2 \bs{\hat{e}}_2$.

\ifsolution
\underline{Solution:}

From above we have that:
\be
d\bs{x}^{(1)}=\bs{F} d\bs{X}^{(1)} \textrm{ and } d\bs{x}^{(2)}=\bs{F} d\bs{X}^{(2)}
\ee
and:
\begin{align}
\label{eq:sol1}
d\bs{x}^{(1)} \cdot d\bs{x}^{(2)} &=ds_1ds_2 \cos \theta= (\bs{F} d\bs{X}^{(1)})\cdot (\bs{F} d\bs{X}^{(2)}) \\
\nonumber &=d\bs{X}^{(1)} \cdot (\bs{F}^T\bs{F})d\bs{X}^{(2)}=d\bs{X}^{(1)}\cdot \bs{C} d\bs{X}^{(2)}
\end{align}
where $ds_1,ds_2$ are the deformed lengths which in the problem  at hand  are given as above by:
\be
(ds_1)^2 =C_{11} (dS_1)^2\to ds_1=3 dS_1
\ee
and:
 \be
(ds_2)^2= C_{22} (dS_2)^2\to ds_2=2 dS_2
\ee
Hence for \refeq{eq:sol1} we have:
\be
ds_1ds_2 \cos \theta = dS_1 dS_2 C_{12} \to \cos \theta =C_{12} \frac{dS_1}{ds_1} \frac{dS_2}{ds_2}=0
\ee
i.e. the angle $\theta$ will remain equal to $90^o$ even after deformation.
\fi

\ei



\item (\textbf{Exam 2013}) 
Consider the unit square OABC where side OA is at a $45^o$ angle with direction $\bs{\hat{e}}_1$ as in Figure
\ref{fig:square}. It undergoes the following deformation:
\be
x_1=kX_2+X_1, \quad x_2=kX_1+X_2, \quad x_3=X_3
\ee
\begin{figure}[ht]
 \centering
 \psfrag{O}{O}
  \psfrag{A}{A}
 \psfrag{C}{C}
 \psfrag{B}{B}
 \psfrag{e1}{$\bs{\hat{e}}_1, X_1, x_1$}
 \psfrag{e2}{$\bs{\hat{e}}_2, X_2,x_2$}
 \includegraphics[height=3cm]{FIGURES/strecch_square.eps}
 \caption{Problem configuration}
 \label{fig:square}
\end{figure}

\bi
\item Find the  length of the sides OA and OC in the deformed state
\item Find the angle between sides OA and OC in the deformed state

\ifsolution
\underline{Solution:}
\begin{align}
&\vec O = (0, \quad 0),~ &&\vec O'= (0, \quad 0), \\ \nonumber
&\vec A = \left(1/\sqrt{2}, \quad 1/\sqrt{2}\right),~ &&\vec A'= \left(\frac{k+1}{\sqrt{2}}, \quad
\frac{k+1}{\sqrt{2}}\right), \\ \nonumber &\vec C = \left(-\frac{1}{\sqrt{2}}, \quad \frac{1}{\sqrt{2}}\right), &&\vec
C' = \left(\frac{k - 1}{\sqrt{2}}, \quad \frac{1 - k}{\sqrt{2}}\right).
\end{align}
Lengths:
\be
\left|\vec O' - \vec A'\right| = \left|\vec A'\right| = \left| k + 1 \right|, \qquad \left|\vec O' -\vec C'\right| =
\left|\vec C'\right| = \left|k - 1\right|.
\ee
Angle:
\be
\left|\vec A'\right|\cdot\left|\vec C'\right| \cos \theta =\vec A' \cdot \vec C' = \frac{1}{2}\left(k^2 - 1 + 1 -
k^2\right) = 0,
\ee
i.e. the sides $O'A', O'C'$ stay perpendicular in the deformed state.
\fi
\ei



\newpage
\item (\textbf{Exam 2013}) 
The cylinder shown in  Figure \ref{fig:cylinder} undergoes the following deformation:
\be
\begin{array}{l}
x_1 = \mu (X_1 \cos (\beta X_3) + X_2 \sin (\beta X_3) ) \\
x_2 = \mu (-X_1 \sin (\beta X_3) + X_2 \cos (\beta X_3)) \\
x_3 = \nu X_3
\end{array}
\ee
where $\mu, \beta, \nu$ are constants. 
\begin{figure}[ht]
 \centering
 \psfrag{x1}{$X_1$}
  \psfrag{x2}{$X_2$}
 \psfrag{x3}{$X_3$}
  \psfrag{L}{$L$}
   \psfrag{f}{$r$}
 \includegraphics[width=4cm,height=4cm]{FIGURES/cylinder2.eps}
 \caption{Problem configuration}
 \label{fig:cylinder}
\end{figure}
\bi
\item Determine the relationship between these constants if the deformation corresponds to an incompressible medium.

\ifsolution
\underline{Solution:}

An infinitesimal volume $dV$ becomes $dv$ after deformation where:
 \be
 dv= \det(\bs{F}) dV
 \ee
 Hence $\det(\bs F)=1$ everywhere if the material is incomressible (i.e. its volume does not change).
 We have that:
 \be
 [\bs{F}]=\left[\begin{array}{lll} \mu \cos (\beta X_3) & \mu \sin (\beta X_3) & \mu \beta (-X_1 \sin (\beta X_3)+X_2
 \cos (\beta X_3)) \\
 -\mu \sin (\beta X_3) & \mu \cos (\beta X_3) &  \mu \beta (-X_1 \cos (\beta X_3)-X_2 \cos (\beta X_3))  \\
 0 & 0 & \nu \end{array} \right]
\ee
Hence $\det(\bs F)=\nu \mu^2$. As a result as long as $\nu \mu^2=1$ the motion is incompressible (for any $\beta$).
\fi
\item Determine the deformed length $l$ of an element on the lateral surface which has unit length and is parallel to
the cylinder axis $X_3$ in the reference configuration. Your final expression could be in terms of the length $L$, the
radius $r$ and the constants $\mu, \beta, \nu$.

\ifsolution
\underline{Solution:}

Consider such a line on the lateral surface and a point P along this line with coordinates $(X_1,X_2,X_3)$. Since $P$
is on the lateral surface:
\be
X_1^2+X_2^2=r^2
\label{eq:cyl}
\ee
Consider also a fiber $d\bs{X}$ originating from $P$ with infinitesimal length $dS$ and in the direction of $X_3$ i.e.:
\be
[d\bs{X}]=dS\left[ \begin{array}{l} 0 \\ 0 \\ 1 \end{array} \right]
\ee
The fiber becomes $d\bs{x}$ is the deformed state and:
\be
[d\bs{x}]=[\bs{F}][d\bs{X}]=dS\left[ \begin{array}{c} \mu \beta (-X_1 \sin (\beta X_3)+X_2 \cos (\beta X_3)) \\ \mu
\beta (-X_1 \cos (\beta X_3)-X_2 \cos (\beta X_3))  \\ \nu \end{array} \right]
\ee
If $ds$ is the deformed infinitesimal length of $d\bs{x}$ then:
\be
\begin{array}{ll} 
ds^2 &=dS^2 ( \mu^2 \beta^2 ((-X_1 \sin (\beta X_3)+X_2 \cos (\beta X_3))^2 \\ &~~+(-X_1 \cos (\beta X_3)-X_2 \cos
(\beta X_3))^2)+\nu^2) \\
&= dS^2 (  \mu^2 \beta^2 ( X_1^2+X_2^2)+\nu^2) \\
& =dS^2(  \mu^2 \beta^2 r^2+\nu^2)
\end{array}
\ee
due to \refeq{eq:cyl}. Hence the length will increase by a factor $\sqrt{\mu^2 \beta^2 r^2+\nu^2}$. This is independent
of the point $P$ along the line we consider and as a result the new length $l$ of the unit undeformed element will be:
\be
l=\sqrt{\mu^2 \beta^2 r^2+\nu^2}
\ee
\fi
\ei

\item  Consider the motion given by:
\be
\bs{x}=\bs{X}+kX_1\bs{\hat{e}}_1
\ee
Let $d\bs{X}^{(1)} = \frac{dS_1}{\sqrt{2}}(\bs{\hat{e}}_1+\bs{\hat{e}}_2)$ and $d\bs{X}^{(2)} =
\frac{dS_2}{\sqrt{2}}(-\bs{\hat{e}}_1 + \bs{\hat{e}}_2)$ be differential material elements in the undeformed
configuration.
\bi
\item Find the deformed elements $d\bs{x}^{(1)}$ and $d\bs{x}^{(2)}$ with lengths $ds_1$ and $ds_2$ respectively.

\ifsolution
\underline{Solution:}

The deformation gradient for the motion above is given by:
\be
[\bs{F}] = \bmat 1+k & 0 & 0 \\ 0 & 1 & 0 \\ 0 & 0 & 1 \emat
\ee

As discussed in the previous problem as well:
\be
d\bs{x}^{(1)}=\bs{F}d\bs{X}^{(1)} \to [d\bs{x}^{(1)}] = \left[\begin{array}{lll}  1+k & 0 & 0 \\ 0 & 1 & 0 \\ 0 & 0 & 1                
               \end{array}
\right] \frac{dS_1}{\sqrt{2}} \left[\begin{array}{l} 1 \\1 \\0 \end{array} \right] =\frac{dS_1}{\sqrt{2}} 
\left[\begin{array}{l} 1+k \\1 \\0 \end{array} \right]
\ee
and:
\be
ds_1=|d\bs{x}^{(1)}|=\frac{dS_1}{\sqrt{2}} \sqrt{(1+k)^2+1}
\ee

Similarly:
\be
d\bs{x}^{(2)}=\bs{F}d\bs{X}^{(2)} \to [d\bs{x}^{(2)}] = \left[\begin{array}{lll}  1+k & 0 & 0 \\ 0 & 1 & 0 \\ 0 & 0 & 1                
               \end{array}
\right] \frac{dS_1}{\sqrt{2}} \left[\begin{array}{l} -1 \\1 \\0 \end{array} \right] =\frac{dS_1}{\sqrt{2}} 
\left[\begin{array}{l} -(1+k) \\1 \\0 \end{array} \right]
\ee
and:
\be
ds_2=|d\bs{x}^{(2)}|=\frac{dS_2}{\sqrt{2}} \sqrt{(1+k)^2+1}
\ee
\fi
\item  Evaluate the stretches of these elements $ds_1/dS_1$ and $ds_2/dS_2$ and the change in the angle between them.

\ifsolution
\underline{Solution:}

From the discussion in the previous problem we have:
\be
|\frac{ds_1}{dS_1}|=\sqrt{\frac{(1+k)^2+1}{2}}
\ee
\be
|\frac{ds_2}{dS_2}|=\sqrt{\frac{(1+k)^2+1}{2}}
\ee
Furthermore the right Cauchy-Green tensor $\bs{C}$ is:
\be
[\bs{C}]=\left[\begin{array}{lll}  (1+k)^2 & 0 & 0 \\ 0 & 1 & 0 \\ 0 & 0 & 1                
               \end{array}
\right]
\ee
and:
\be
\begin{array}{ll}
ds_1ds_2\cos \theta & =d\bs{X}^{(1)}\cdot \bs{C}d\bs{X}^{(2)} \\
& =\frac{dS_1}{\sqrt{2}}  \left[\begin{array}{lll} 1 &1 &0 \end{array} \right]  \left[\begin{array}{lll} (1+k)^2 & 0 &
0 \\ 0 & 1 & 0 \\ 0 & 0 & 1
               \end{array}
\right] \frac{dS_2}{\sqrt{2}}  \left[\begin{array}{l} -1 \\1 \\0 \end{array} \right]\\
&=\frac{dS_1dS_2}{2}(-(1+k)^2+1)
\end{array}
\ee
\fi
\item Evaluate part (b) for $k = 1$ and $k = 0.01$.

\ifsolution
\underline{Solution:}

For $k = 1$:
\be
|\frac{ds_1}{dS_1}|=\sqrt{\frac{(1+k)^2+1}{2}}=\sqrt{\frac{5}{2}}=1.581
\ee
\be
|\frac{ds_2}{dS_2}|=\sqrt{\frac{(1+k)^2+1}{2}}=\sqrt{\frac{5}{2}}=1.581
\ee
and:
\be
ds_1ds_2\cos \theta=\frac{dS_1dS_2}{2}(-(1+k)^2+1)=-\frac{3 dS_1dS_2}{2} \to cos\theta =-0.6
\ee
For $k=0.01$:
\be
|\frac{ds_1}{dS_1}|=\sqrt{\frac{(1+k)^2+1}{2}}=1.005
\ee
\be
|\frac{ds_2}{dS_2}|=\sqrt{\frac{(1+k)^2+1}{2}}=1.005
\ee
and:
\be
ds_1ds_2\cos \theta=\frac{dS_1dS_2}{2}(-(1+k)^2+1) = -\frac{3 dS_1dS_2}{2} \to cos\theta =-0.00995
\ee
\fi
\item Compare the results of the previous part to that predicted by the infinitesimal  strain tensor $\bs{\epsilon}$.

\ifsolution
\underline{Solution:}
The Lagrangean strain tensor $\bs{E}$ relates to  the right Cauchy-Green tensor $\bs{C}$ as:
\be
\bs{E}=\frac{1}{2}(\bs{C}-\bs{I}) \leftrightarrow \bs{C}=2\bs{E}+\bs{I}
\ee
We can therefore use $[\bs{E}]=\left[\begin{array}{lll}  \frac{(1+k)^2-1}{2} & 0 & 0 \\ 0 & 0 & 0 \\ 0 & 0 & 0                
               \end{array}
\right]$ to derive the results above, e.g.:
\be
(ds_1)^2=d\bs{X}^{(1)}\cdot (2\bs{E}+\bs{I}) d\bs{X}^{(1)}, \textrm{ or } (ds_1)^2=d\bs{X}^{(2)}\cdot (2\bs{E}+\bs{I})
d\bs{X}^{(2)}
\ee
and:
\be
ds_1ds_2\cos \theta=d\bs{X}^{(1)}\cdot (2\bs{E}+\bs{I}) d\bs{X}^{(2)}
\ee
If in the equations above we use instead of $\bs{E}$, the infinitesimal strain tensor $\bs{\epsilon}$ given by:
\be
\bs{\epsilon}=\frac{1}{2}(\bs{F}+\bs{F}^T)-\bs{I} \to [\bs{\epsilon}]=\left[\begin{array}{lll}  k & 0 & 0 \\ 0 & 0 & 0
\\ 0 & 0 & 0
               \end{array}
\right]
\ee
then we obtain:
\be
\begin{array}{ll}
(ds_1)^2& =d\bs{X}^{(1)}\cdot (2\bs{\epsilon}+\bs{I}) d\bs{X}^{(1)} \\
&= \frac{dS_1}{\sqrt{2}} \left[\begin{array}{lll} 1  & 1 & 0 \end{array} \right] \left[\begin{array}{lll}  2k+1 & 0 & 0
\\ 0 & 1 & 0 \\ 0 & 0 & 1 \end{array} \right] \frac{dS_1}{\sqrt{2}} \left[\begin{array}{l} 1 \\1 \\0 \end{array}
\right] \\
& =dS_1^2 (k+1)
\end{array} 
\ee
and:
\be
\begin{array}{ll}
(ds_2)^2& =d\bs{X}^{(2)}\cdot (2\bs{\epsilon}+\bs{I}) d\bs{X}^{(2)} \\
&= \frac{dS_2}{\sqrt{2}} \left[\begin{array}{lll} -1  & 1 & 0 \end{array} \right] \left[\begin{array}{lll}  2k+1 & 0 &
0 \\ 0 & 1 & 0 \\ 0 & 0 & 1
               \end{array}
\right] \frac{dS_2}{\sqrt{2}} \left[\begin{array}{l} -1 \\1 \\0 \end{array} \right] \\
&= dS_2^2 (k+1)
\end{array} 
\ee
and:
\be
\begin{array}{ll}
ds_1ds_2\cos \theta& =d\bs{X}^{(1)}\cdot (2\bs{\epsilon}+\bs{I}) d\bs{X}^{(2)} \\
&= \frac{dS_1}{\sqrt{2}} \left[\begin{array}{lll} 1  & 1 & 0 \end{array} \right] \left[\begin{array}{lll}  2k+1 & 0 & 0
\\ 0 & 1 & 0 \\ 0 & 0 & 1 \end{array} \right] \frac{dS_2}{\sqrt{2}} \left[\begin{array}{l} -1 \\1 \\0 \end{array}
\right] \\
& = -k dS_1 dS_2
\end{array}
\ee

We observe that for $k=1$:
\be
|\frac{ds_1}{dS_1}|=\sqrt{k+1}=\sqrt{2} \ne 1.581
\ee
\be
|\frac{ds_2}{dS_2}|=\sqrt{k+1}=\sqrt{2} \ne 1.581
\ee
and:
\be
ds_1ds_2\cos \theta=-k dS_1dS_2 \to cos\theta =-\frac{1}{2}\ne -0.6
\ee
For $k=0.01$:
\be
|\frac{ds_1}{dS_1}|=\sqrt{k+1}=1.00498 \approx 1.005
\ee
\be
|\frac{ds_2}{dS_2}|=\sqrt{k+1}=1.00498 \approx 1.005
\ee
and:
\be
ds_1ds_2\cos \theta=-k dS_1dS_2 \to cos\theta =-0.0099 \approx -0.00995
\ee

We observe that for small $k$ the infinitesimal strain tensor provides a good approximation. This is because the
magnitude of the displacement gradients $\nabla \bs{U}$ are proportional to $k$ and therefore when $k$ is small, we
expect $\bs{\epsilon}$ to be a good approximation.

Observe also that the expressions for the infinitesimal strain tensor can be obtained by removing any $k^2$ order terms
from the expressions with the Lagrangean strain tensor.
\fi
\ei









\item A motion is said to be irrotational if the spin tensor vanishes. Show that the following velocity field is
irrotational:
\be
\bs{v}=\frac{-x_2 \bs{\hat{e}}_1+x_1\bs{\hat{e}}_2 }{r^2}, \quad \tm{where} \quad r^2 = x_1^2+x_2^2.
\ee

\ifsolution
\underline{Solution:}

The spin tensor $\bs{W}$ is defined as:
\be
\bs{W}=\frac{1}{2}( \nabla \bs{v}-(\nabla \bs{v})^T)
\ee
where $\nabla \bs{v}$ is the velocity gradient (spatial description).
From the velocity field above we have that:
\be
[\bs{v}]=\frac{1}{r^4}\left[\begin{array}{lll}   2x_1x_2 & x_2^2-x_1^2 & 0 \\
                             x_2^2-x_1^2 &-2x_1x_2 & 0 \\ 0 & 0 & 0
                            \end{array}\right]
\ee
from which one can easily verify that:
\be
\bs{W}=\bs{0}.
\ee
\fi

\newpage
\item (\textbf{Exam 2014})  Figure \ref{fig:glacier} depicts the undeformed and deformed state of the same material
body. In order to track the deformation a triangle ABC was drawn on the material. In the undeformed state, the length
of all the sides was equal to $a_0=100~mm$. In the deformed state, the triangle has deformed and the length of its
sides are $a_1=80~mm$ (side AB), $a_2=110~mm$ (side BC), $a_3=120~mm$ (side CA).
Assuming that:
\bi
\item the deformation is homogeneous i.e. the deformation gradient $\bs{F}$ is constant in space
\item the out-of-plane deformation is zero i.e. everything happens in the $X_1-X_2$ plane
\item the deformations are large, i.e. the small deformations assumption (infinitesimal strain tensor) does NOT hold
\ei
You are asked to find the non-zero components of the Lagrangean strain tensor $\bs{E}=\frac{1}{2}(\bs{F}^T
\bs{F}-\bs{I})$.  It suffices to find and clearly indicate the necessary equations for the determination of the
components of $\bs{E}$ in order to get full credit.

\begin{figure}[!h]
\psfrag{before}{\underline{Undeformed}}
\psfrag{after}{\underline{Deformed}}
\psfrag{A}{A}
\psfrag{B}{B}
\psfrag{C}{C}
\psfrag{a0}{$a_0$}
\psfrag{a1}{$a_1$}
\psfrag{a2}{$a_2$}
\psfrag{a3}{$a_3$}

\psfrag{x1}{$X_1$}
\psfrag{x2}{$X_2$}
\centering
  \includegraphics[height=5cm,width=.75\textwidth]{FIGURES/glacier.eps}
 \caption{Problem configuration}
 \label{fig:glacier}
\end{figure}

\ifsolution
\underline{Solution:}

\begin{align}
d\bs X^{(1)} &= 
\begin{pmatrix}
-\frac{1}{2} a_0\\
-\frac{\sqrt{3}}{2} a_0
\end{pmatrix}
&&\textit{Side $AB$}, \\
d\bs X^{(2)} &= 
\begin{pmatrix}
 a_0\\
0
\end{pmatrix}
&&\textit{Side $BC$}, \\
d\bs X^{(3)} &= 
\begin{pmatrix}
-\frac{1}{2} a_0\\
\frac{\sqrt{3}}{2} a_0
\end{pmatrix}
&&\textit{Side $CA$}.
\end{align}

\begin{align}
a_i^2 &= d\bs X^{(i)T} \bs F^T \bs F d \bs X^{(i)} = d\bs X^{(i)T} \bs C d \bs X^{(i)} && \textit{$i$ not summed,} \\
a_1^2 &= dX_i^{(1)} C_{ij}dX_j^{(1)} = \frac{a_0^2}{4} C_{11} + \frac{\sqrt{3}}{2} a_0^2 C_{12} + \frac{3}{4} a_0^2 C_{22}, && \\
a_2^2 &= dX_i^{(2)} C_{ij} dX_j^{(2)} = a_0^2 C_{11}, && \\
a_3^2 &=dX_i^{(3)} C_{ij} dX_j^{(3)} = \frac{a_0^2}{4}C_{11} - \frac{\sqrt{3}}{2}a_0^2 C_{12} + \frac{3}{4}a_0^2 C_{22}.
\end{align}

We thus have 3 linear independent equations with 3 unknowns $C_{11},C_{12}, C_{22}$. Solving for $\bs C$, we can compute $\bs E$ by
\be
\bs C = 2\bs E + \bs I.
\ee
\fi

\item (\tbf{Exam 2015}) A velocity field is defined in terms of the spatial coordinates and time by the
equations:
\be
v_1 = 2tx_1 \sin x_3, \qquad v_2 = 2tx_2 \cos x_2, \qquad v_3.
\ee
At the point (1,1,0) and at time t = 1, determine:
\bi
\item the rate of deformation tensor $\bs D = \frac{1}{2}(\nabla^x \bs v + (\nabla^x \bs v)^T)$
\item the time rate of extension per unit length in the direction $(\hat{\bs e}_1 + \hat{\bs e}_2 + \hat{\bs e}_3)/\sqrt{3}$
\ei

\ifsolution
\underline{Solution:}
\be
D_{ij} = \frac{1}{2}(v_{i,j} + v_{j,i}).
\ee

\begin{align}
&v_{1,1} = 2t\sin x_3 &&v_{2,1} = 0 &&&v_{3,1} = 0 \\ \nonumber
&v_{1,2} = 0 &&v_{2,2} = 2t\cos x_2 - 2tx_2\sin x_2 &&&v_{3,2} = 0 \\ \nonumber
&v_{1,3} = 2tx_1 \cos x_3 &&v_{2,3} = 0 &&&v_{3,3} = 0,
\end{align}
Hence
\be
\bs D = \begin{pmatrix}
2t\sin x_3 & 0 & tx_1 \cos x_3 \\
0 & 2t\cos x_2 - 2t x_2 \sin x_2 & 0 \\
t x_1 \cos x_3 & 0 & 0
\end{pmatrix},
\ee
\be
\bs D(\bs x = (1,1,0), t = 1) = \begin{pmatrix}
0 & 0 & 1 \\
0 & 2\cos 1 - 2\sin 1 & 0 \\
1 & 0 & 0
\end{pmatrix}.
\ee
Lecture:

For $d\bs x = \bs n dx$ with $\bs n$ the normalized direction vector,
\be
\frac{1}{dx}\frac{d}{dt}(dx) = \bs n \cdot \bs D \bs n = 2 + 2\cos 1 - 2\sin 1.
\ee
\fi

\end{enumerate}



\end{document}
