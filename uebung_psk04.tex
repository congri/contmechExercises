\documentclass{article}
\usepackage{nips07submit_e,times}
\usepackage{color}
%\documentstyle[nips07submit_09,times]{article}

\title{\underline{MSE - WS 2014/15}\\
Continuum Mechanics}


\author{
Instructor: P.S. Koutsourelakis  \\
\texttt{p.s.koutsourelakis@tum.de} \\
}


% The \author macro works with any number of authors. There are two commands
% used to separate the names and addresses of multiple authors: \And and \AND.
%
% Using \And between authors leaves it to \LaTeX{} to determine where to break
% the lines. Using \AND forces a linebreak at that point. So, if \LaTeX{}
% puts 3 of 4 authors names on the first line, and the last on the second
% line, try using \AND instead of \And before the third author name.


%
%%%%%%%%%%%%%%%%%%%%%%%%%%%%%%%%%%%%%%%%%%%%%%%%%%%%%%%
\usepackage{amsmath,amssymb}
\usepackage{hyperref} 

\usepackage[dvips]{psfrag,graphicx}


\linespread{1.6}
\newcommand{\ee}{\end{equation}}
\newcommand{\be}{\begin{equation}}
\newcommand{\ec}{\end{center}}
\newcommand{\bc}{\begin{center}}
\newcommand{\eea}{\end{eqnarray}}
\newcommand{\bea}{\begin{eqnarray}}
\newcommand{\bd}{\begin{description}}
\newcommand{\ed}{\end{description}}
\newcommand{\bi}{\begin{itemize}}
\newcommand{\ei}{\end{itemize}}
\newcommand{\pa}{\partial}
\newcommand{\bs}{\boldsymbol}
\def\RR{ \mathbb R}
\newcommand{\refeq}[1]{Equation (\ref{#1})}
%%%%%%%%%%%%%%%%%%%%%%%%%%%%%%%%%%%%%%%%%%%%%%%%%%%%%%
%

\begin{document}

\makeanontitle

%\begin{abstract}
%\input{abstract}
%\end{abstract}

\section*{\"Ubung04}


\begin{enumerate}
\item (\textbf{Exam 2013}) 
For an isotropic linearly elastic solid under small deformations, the stress-strain relation is $\sigma_{ij}=\lambda \delta_{ij} \epsilon_{kk}+2\mu \epsilon_{ij}$ where $\lambda, \mu$ are Lam\'e's constants.
If the  displacement field is:
\be
u_1= k x_3 x_2, \quad  u_2= kx_3 x_1, \quad  u_3= k (x_1x_2+x_3^2)%, \quad k=10^{-4}
\ee
\bi
\item  Find the stress components.
\item  In the absence of body forces, check whether  the state of stress corresponds to an
equilibrium stress field.
\ei


\item  An isotropic elastic body with $E=  207~ GPa$, $\mu= 79.2~GPa$,  has a uniform state of stress given by:
\be
[\bs{\sigma}]=\left[\begin{array}{ccc} 100 & 40 & 60 \\ 40 & -200 & 0 \\ 60 & 0 & 200 \end{array} \right] ~MPa
\ee
\bi
\item What are the strain components?
\item  What is the total change of volume for a five-centimeter cube of the material?
\ei

%Mase 6.4
\item  For an isotropic elastic medium, express the strain energy density function $U$ in terms of:
\bi
\item the components of $\epsilon_{ij}$
\item the components of $\sigma_{ij}$
\item the invariants of $\epsilon_{ij}$
\ei



\item  Given the following displacement field in an isotropic linearly elastic solid:
\be
u_1= k x_3 x_2, \quad  u_2= kx_3 x_1, \quad  u_3= k (x_1^2-x_2^2), \quad k=10^{-4}
\ee
\bi
\item  Find the stress components, and
\item  In the absence of body forces, check whether  the state of stress corresponds to an
equilibrium stress field?
\ei

\item  Let the stress field for a continuum be given by:
\be
[\bs{\sigma}]=\left[\begin{array}{ccc} x_1+x_2 & \sigma_{12}(x_1,x_2) & 0 \\  \sigma_{12}(x_1,x_2) & x_1-x_2 & 0 \\ 0 & 0 & x_2 \end{array} \right]
\ee
If the equilibrium equations are satisfied in the absence of body forces and if the stress vector on the plane perpendicular to  $\bs{\hat{e}}_1$ at $x_1 = 1$ is given by $\bs{t} = (1 + x_2) \bs{\hat{e}}_1 + (6 -x_2 )\bs{\hat{e}}_2$, determine $\sigma_{12}(x_1,x_2)$.
%Answer: σ12 = x1 – x2 + 5

\item  A composite bar, formed by welding two slender, cylindrical bars of equal length and equal cross-sectional area, is
loaded by an axial load P as shown in Figure \ref{fig:543}. If the Young's moduli of the two portions are $E_1$ and $E_2$,  find how the applied force is distributed between the two halves.
\begin{figure}[!h]
\centering
 \includegraphics[width=.75\textwidth]{FIGURES/problem5.43.eps}
\caption{Uniaxial extension of a prismatic bar}
\label{fig:543}
\end{figure}

\item Consider the case of pure bending discussed in class (Figure \ref{fig:pb}) with moment couple $\bs{M}_R=-\bs{M}_L=M\hat{\bs{e}}_2$. The  $x_1$ axis is passing
through the centroids of all the cross-sections  and the  $x_2$ and $x_3$ axes are the principal axes. The beam is subjected to the following constraints: 
\bi
\item  The origin $(0,0,0)$ is fixed, 
\item $\frac{\pa u_3}{\pa x_2}=0$ at the origin
\item  the centroid at the right end section can only move horizontally in the $x_1$ direction. \ei
\bi
\item Obtain the displacement field and show that every plane cross-section remains a plane after bending and 
\item Obtain the deformed shape of the centroidal line of the beam, regarded as the deflection of the beam.
\ei
\begin{figure}[!h]
\centering
\psfrag{x1}{$x_1$}
\psfrag{x2}{$x_2$}
\psfrag{x3}{$x_3$}
\psfrag{ML}{$M_L$}
\psfrag{MR}{$M_R$}
\psfrag{l}{$l$}
 \includegraphics[width=.75\textwidth]{FIGURES/bending.eps}
\caption{Pure bending of a prismatic bar}
\label{fig:pb}
\end{figure}


\end{enumerate}



\end{document}
