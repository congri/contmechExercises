\documentclass{article}
\usepackage{nips07submit_e,times}
\usepackage{color}
%\documentstyle[nips07submit_09,times]{article}

\title{\underline{MSE - WS 2014/15}\\
Continuum Mechanics}


\author{
Instructor: P.S. Koutsourelakis  \\
\texttt{p.s.koutsourelakis@tum.de} \\
}


% The \author macro works with any number of authors. There are two commands
% used to separate the names and addresses of multiple authors: \And and \AND.
%
% Using \And between authors leaves it to \LaTeX{} to determine where to break
% the lines. Using \AND forces a linebreak at that point. So, if \LaTeX{}
% puts 3 of 4 authors names on the first line, and the last on the second
% line, try using \AND instead of \And before the third author name.


%
%%%%%%%%%%%%%%%%%%%%%%%%%%%%%%%%%%%%%%%%%%%%%%%%%%%%%%%
\usepackage{amsmath,amssymb}
\usepackage{hyperref} 

\usepackage[dvips]{psfrag,graphicx}


\linespread{1.6}
\newcommand{\ee}{\end{equation}}
\newcommand{\be}{\begin{equation}}
\newcommand{\ec}{\end{center}}
\newcommand{\bc}{\begin{center}}
\newcommand{\eea}{\end{eqnarray}}
\newcommand{\bea}{\begin{eqnarray}}
\newcommand{\bd}{\begin{description}}
\newcommand{\ed}{\end{description}}
\newcommand{\bi}{\begin{itemize}}
\newcommand{\ei}{\end{itemize}}
\newcommand{\pa}{\partial}
\newcommand{\bs}{\boldsymbol}
\def\RR{ \mathbb R}
\newcommand{\refeq}[1]{Equation (\ref{#1})}
%%%%%%%%%%%%%%%%%%%%%%%%%%%%%%%%%%%%%%%%%%%%%%%%%%%%%%
%

\begin{document}

\makeanontitle

%\begin{abstract}
%\input{abstract}
%\end{abstract}

\section*{\"Ubung05}



\begin{enumerate}

%Lai 5.63

\item   Consider the Airy stress function $\phi=  a_1 x_1^2+   a_2 x_1 x_2 +  a_3 x_2^2$ . (a) Verify that it satisfies the biharmonic equation. (b) Determine the in-plane stresses $\sigma_{11}, \sigma_{12}$ and $\sigma_{22}$. (c) Determine and sketch the tractions
on the four rectangular boundaries $x_1 = 0$,  $x_1 = b$,  $x_2=0$, $x_2= c$. (d) As a plane strain solution, determine $\sigma_{13}$, $\sigma_{23}$, $\sigma_{33}$ and all the strain components. (e) As a plane stress solution, determine $\sigma_{13}$, $\sigma_{23}$, $\sigma_{33}$
and all the strain components.

%Lai 5.65
\item   Consider the Airy stress function $\phi=  a (x_1^4-x_2^4)$ . (a) Verify that it satisfies the biharmonic equation. (b) Determine the in-plane stresses $\sigma_{11}, \sigma_{12}$ and $\sigma_{22}$. (c) Determine and sketch the tractions
on the four rectangular boundaries $x_1 = 0$,  $x_1 = b$,  $x_2=0$, $x_2= c$. (d) As a plane strain solution, determine $\sigma_{13}$, $\sigma_{23}$, $\sigma_{33}$ and all the strain components. (e) As a plane stress solution, determine $\sigma_{13}$, $\sigma_{23}$, $\sigma_{33}$
and all the strain components.

%Lai 5.66
\item   Consider the Airy stress function $\phi=  a  x_1 x_2^2+  x_1 x_2^3$ . (a) Verify that it satisfies the biharmonic equation. (b) Determine the in-plane stresses $\sigma_{11}, \sigma_{12}$ and $\sigma_{22}$. (c)   Determine the condition necessary for the traction at $x_2 = c$ to vanish, (d) determine  the tractions on the remaining boundaries $x_1 = 0$,  $x_1 = b$,  $x_2=0$. 

\begin{figure}[!h]
\centering
\psfrag{x1}{$x_1$}
\psfrag{x2}{$x_2$}
\psfrag{h}{$h$}
\psfrag{l}{$l$}
\psfrag{1}{$1$}
\psfrag{p}{$P$}
\includegraphics[width=0.65\textwidth]{FIGURES/psbeam2.eps}
\label{fig:cantips}
\caption{Problem 4}
\end{figure}
%Lai 5.67
 \item Obtain the in-plane displacement components for the plane stress solution for the cantilever beam from the following strain-displacement relations:
\be
\epsilon_{11}=\frac{ \pa u_1}{\pa x_1}=-\frac{P x_1x_2}{E I}, \quad \epsilon_{22}=\frac{ \pa u_2}{\pa x_2}=\frac{\nu P x_1x_2}{E I}, \quad \epsilon_{12}=-\frac{P }{4 \mu  I} (h^2-x_2^2)
\ee

\item (\textbf{Exam 2014}) 
The isotropic, elastic block ABC  ($E,\nu$ are given) in Figure \ref{fig:dam} is under plane strain conditions. It is subjected to hydrostatic pressure  along AB (as shown in the Figure) and is free along AC.
Given the Airy stress function:
\be
\phi(x,y)= \frac{1}{6} a_1 x^3+ \frac{1}{2} a_2 x^2 y+ \frac{1}{2} a_3x y^2 +\frac{1}{6}a_4 y^3
\ee
\bi
\item Determine  {\em all} the stress components  inside ABC using the traction boundary conditions above. You can ignore the block's own weight.
\item What tractions must be applied along BC in order for the block to be in equilibrium? Please provide a sketch with the appropriate values.

\ei
\begin{figure}[th]
\psfrag{l}{$l$}
\psfrag{p}{$\rho g y$}
\psfrag{a}{$A$}
\psfrag{B}{$B$}
\psfrag{C}{$C$}
\psfrag{x}{$x$}
\psfrag{y}{$y$}
\centering \includegraphics[width=0.45\textwidth,height=4cm]{FIGURES/dam.eps}
\caption{ Problem configuration}
\label{fig:dam}
\end{figure}
\newpage

\end{enumerate}

\end{document}
