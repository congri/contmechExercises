\documentclass{article}
\usepackage{nips07submit_e,times}
\usepackage{color}
%\documentstyle[nips07submit_09,times]{article}

\title{\underline{MSE - WS 2015/16}\\
Continuum Mechanics}


\author{
Instructor: P.S. Koutsourelakis  \\
\texttt{p.s.koutsourelakis@tum.de} \\
}


% The \author macro works with any number of authors. There are two commands
% used to separate the names and addresses of multiple authors: \And and \AND.
%
% Using \And between authors leaves it to \LaTeX{} to determine where to break
% the lines. Using \AND forces a linebreak at that point. So, if \LaTeX{}
% puts 3 of 4 authors names on the first line, and the last on the second
% line, try using \AND instead of \And before the third author name.


%
%%%%%%%%%%%%%%%%%%%%%%%%%%%%%%%%%%%%%%%%%%%%%%%%%%%%%%%
\usepackage{amsmath,amssymb}
\usepackage{hyperref} 

\usepackage[dvips]{psfrag,graphicx}


\linespread{1.6}
\newcommand{\ee}{\end{equation}}
\newcommand{\be}{\begin{equation}}
\newcommand{\ec}{\end{center}}
\newcommand{\bc}{\begin{center}}
\newcommand{\eea}{\end{eqnarray}}
\newcommand{\bea}{\begin{eqnarray}}
\newcommand{\bd}{\begin{description}}
\newcommand{\ed}{\end{description}}
\newcommand{\bi}{\begin{itemize}}
\newcommand{\ei}{\end{itemize}}
\newcommand{\pa}{\partial}
\newcommand{\bs}{\boldsymbol}
\def\RR{ \mathbb R}
\newcommand{\refeq}[1]{Equation (\ref{#1})}
%%%%%%%%%%%%%%%%%%%%%%%%%%%%%%%%%%%%%%%%%%%%%%%%%%%%%%
%

\begin{document}

\makeanontitle

%\begin{abstract}
%\input{abstract}
%\end{abstract}

\section*{\"Ubung 05}


\begin{enumerate}

%Lai 5.63

\item   Consider the Airy stress function $\phi=  a_1 x_1^2+   a_2 x_1 x_2 +  a_3 x_2^2$ . (a) Verify that it satisfies the biharmonic equation. (b) Determine
the in-plane stresses $\sigma_{11}, \sigma_{12}$ and $\sigma_{22}$. (c) Determine and sketch the tractions on the four rectangular boundaries $x_1 = 0$,  $x_1 = b$,  $x_2=0$, $x_2= c$. (d) As a plane strain solution, determine $\sigma_{13}$, $\sigma_{23}$, $\sigma_{33}$ and all the strain components. (e) As a plane stress solution, determine $\sigma_{13}$, $\sigma_{23}$, $\sigma_{33}$
and all the strain components.

\underline{Solution:} \\
\bi
\item The biharmonic $\nabla^4 \phi=\frac{\pa^4  \phi}{\pa x_1^4}+2\frac{\pa^4  \phi}{\pa x_1^2 \pa x_2^2}+\frac{\pa^4  \phi}{\pa x_2^4}=0$ is satisfied as the fourth order derivatives are 0.

\item 
\be
\begin{array}{l}
\sigma_{11}=\frac{\pa^2  \phi}{\pa x_2^2}=2a_3 \\
 \sigma_{22}=\frac{\pa^2  \phi}{\pa x_1^2}=2a_1 \\
\sigma_{12}=-\frac{\pa^2  \phi}{\pa x_1 \pa x_2}=-a_2
\end{array}
\ee

\item On the four boundaries we have:
\bi
\item At $x_1=0$, $\bs{n}=-\bs{\hat{e}}_1$, the boundary tractions are:
\be
\bs{\sigma}|_{x_1=0} \bs{n}=-\sigma_{11}(x_1=0)\bs{\hat{e}}_1-\sigma_{12}(x_1=0)\bs{\hat{e}}_2=-2a_3\bs{\hat{e}}_1+a_2\bs{\hat{e}}_2
\ee
\item At $x_1=b$, $\bs{n}=\bs{\hat{e}}_1$, the boundary tractions are:
\be
\bs{\sigma}|_{x_1=b} \bs{n}=\sigma_{11}(x_1=b)\bs{\hat{e}}_1+\sigma_{12}(x_1=b)\bs{\hat{e}}_2=2a_3\bs{\hat{e}}_1-a_2\bs{\hat{e}}_2
\ee
\item At $x_2=0$, $\bs{n}=-\bs{\hat{e}}_2$, the boundary tractions are:
\be
\bs{\sigma}|_{x_2=0} \bs{n}=
-\sigma_{12}(x_2=0)\bs{\hat{e}}_1-\sigma_{22}(x_2=0)\bs{\hat{e}}_2=a_2\bs{\hat{e}}_1-2a_1\bs{\hat{e}}_2
\ee
\item At $x_2=c$, $\bs{n}=\bs{\hat{e}}_2$, the boundary tractions are:
\be
\bs{\sigma}|_{x_2=c} \bs{n}=\sigma_{12}(x_2=c)\bs{\hat{e}}_1+\sigma_{22}(x_2=c)\bs{\hat{e}}_2=-a_2\bs{\hat{e}}_1+2a_1\bs{\hat{e}}_2
\ee

\ei


\item Under plane strain and for an isotropic material, the stresses are:
\be
\sigma_{33}=\nu(\sigma_{11}+\sigma_{22})=\nu (2a_3+2a_1), \quad \sigma_{13}=\sigma_{23}=0.
\ee
%Edited by C. Grigo, 25/01/16
For the strains, we have
\be
\epsilon_{13} = \epsilon_{23} = \epsilon_{33} = 0 \qquad \textrm{(by definition)},
\ee
and from
\be
\epsilon_{ij} = \frac{1}{E}\left(\left(1 + \nu\right)\sigma_{ij} - \nu\sigma_{kk}\delta_{ij}\right),
\label{strain}
\ee
we get
\begin{align}
\epsilon_{12} &= -\frac{a_2}{E}(1 + \nu), \\
\epsilon_{11} &= \frac{1}{E}\left(2a_3 - 2\nu a_1 - 2\nu^2(a_1 + a_3) \right), \\
\epsilon_{22} &= \frac{1}{E}\left(2a_1 - 2\nu a_3 - 2\nu^2 (a_1 + a_3\right).
\end{align}

\item Under plane stress conditions (by definition):
\be
\sigma_{13}=\sigma_{23}=\sigma_{33}=0.
\ee
%Edited by C. Grigo, 25/01/16
Using equation \eqref{strain}, we get
\begin{align}
&\epsilon_{11} = \frac{2}{E}(a_3 - \nu a_1), \qquad \epsilon_{22} =  \frac{2}{E}(a_1 - \nu a_3), \qquad \epsilon_{33} = -\frac{2\nu}{E}(a_1 + a_3) \\ \nonumber
&\epsilon_{12} = -\frac{a_2}{E}(1 + \nu), \qquad \epsilon_{13} = 0, \qquad \epsilon_{23} = 0.
\end{align}

\ei

%Lai 5.65
\item   Consider the Airy stress function $\phi=  a (x_1^4-x_2^4)$ . (a) Verify that it satisfies the biharmonic equation. (b) Determine the in-plane stresses $\sigma_{11}, \sigma_{12}$ and $\sigma_{22}$. (c) Determine and sketch the tractions
on the four rectangular boundaries $x_1 = 0$,  $x_1 = b$,  $x_2=0$, $x_2= c$. (d) As a plane strain solution, determine $\sigma_{13}$, $\sigma_{23}$, $\sigma_{33}$ and all the strain components. (e) As a plane stress solution, determine $\sigma_{13}$, $\sigma_{23}$, $\sigma_{33}$
and all the strain components.


\underline{Solution:} \\
\bi
\item The biharmonic $\nabla^4 \phi=\frac{\pa^4  \phi}{\pa x_1^4}+2\frac{\pa^4  \phi}{\pa x_1^2 \pa x_2^2}+\frac{\pa^4  \phi}{\pa x_2^4}=24a+0-24a=0$ is satisfied.

\item 
\be
\begin{array}{l}
\sigma_{11}=\frac{\pa^2  \phi}{\pa x_2^2}=-12ax_2^2 \\
 \sigma_{22}=\frac{\pa^2  \phi}{\pa x_1^2}=12ax_1^2 \\
\sigma_{12}=-\frac{\pa^2  \phi}{\pa x_1 \pa x_2}=0
\end{array}
\ee


\item On the four boundaries we have:
\bi
\item At $x_1=0$, $\bs{n}=-\bs{\hat{e}}_1$, the boundary tractions are:
\be
\bs{\sigma}|_{x_1=0} \bs{n}=-\sigma_{11}(x_1=0)\bs{\hat{e}}_1-\sigma_{12}(x_1=0)\bs{\hat{e}}_2=12ax_2^2\bs{\hat{e}}_1
\ee
\item At $x_1=b$, $\bs{n}=\bs{\hat{e}}_1$, the boundary tractions are:
\be
\bs{\sigma}|_{x_1=b} \bs{n}=\sigma_{11}(x_1=b)\bs{\hat{e}}_1+\sigma_{12}(x_1=b)\bs{\hat{e}}_2=-12ax_2^2\bs{\hat{e}}_1
\ee
\item At $x_2=0$, $\bs{n}=-\bs{\hat{e}}_2$, the boundary tractions are:
\be
\bs{\sigma}|_{x_2=0} \bs{n}=
-\sigma_{12}(x_2=0)\bs{\hat{e}}_1-\sigma_{22}(x_2=0)\bs{\hat{e}}_2=-12ax_1^2\bs{\hat{e}}_2
\ee
\item At $x_2=c$, $\bs{n}=\bs{\hat{e}}_2$, the boundary tractions are:
\be
\bs{\sigma}|_{x_2=c} \bs{n}=\sigma_{12}(x_2=c)\bs{\hat{e}}_1+\sigma_{22}(x_2=c)\bs{\hat{e}}_2=12ax_1^2\bs{\hat{e}}_2
\ee

\ei


\item Under plane strain and for an isotropic material:
\be
\sigma_{33}=\nu(\sigma_{11}+\sigma_{22})=12a\nu (-x_2^2+x_1^2), \quad \sigma_{13}=\sigma_{23}=0.
\ee
%Edited by C. Grigo 25/01/16
For the strains, we get
\be
\epsilon_{13} = \epsilon_{23} = \epsilon_{33} = 0 \qquad \textrm{(by definition)},
\ee
and from equation \eqref{strain}
\begin{align}
\epsilon_{12} &= \frac{1 + \nu}{E} \sigma_{12} = 0, \\
\epsilon_{11} &= -\frac{12a}{E}\left(x_2^2 + \nu x_1^2 + \nu^2(x_1^2 - x_2^2)\right), \\
\epsilon_{22} &= \frac{12a}{E}\left(x_1^2 + \nu x_2^2 - \nu^2 (x_1^2 - x_2^2)\right).
\end{align}


\item Under plane stress conditions (by definition):
\be
\sigma_{13}=\sigma_{23}=\sigma_{33}=0,
\ee
%Edited by C. Grigo 25/01/16
and for the strains
\begin{align}
\epsilon_{11} &= -\frac{12a}{E}\left(\nu x_1^2 + x_2^2\right), &\epsilon_{22} = \frac{12a}{E}\left(x_1^2 + \nu x_2^2\right), \\
\epsilon_{33} &= \frac{12a\nu}{E}\left(x_2^2 - x_1^2\right), &\epsilon_{12} = 0, \\
\epsilon_{13} &= 0, &\epsilon_{23} = 0.
\end{align}

\ei

%Lai 5.66
\item   Consider the Airy stress function $\phi=  a  x_1 x_2^2+  x_1 x_2^3$ . (a) Verify that it satisfies the biharmonic equation. (b) Determine the in-plane stresses $\sigma_{11}, \sigma_{12}$ and $\sigma_{22}$. (c)   Determine the condition necessary for the traction at $x_2 = c$ to vanish, (d) determine  the tractions on the remaining boundaries $x_1 = 0$,  $x_1 = b$,  $x_2=0$. 



\underline{Solution:} \\

\bi
\item The biharmonic $\nabla^4 \phi=\frac{\pa^4  \phi}{\pa x_1^4}+2\frac{\pa^4  \phi}{\pa x_1^2 \pa x_2^2}+\frac{\pa^4  \phi}{\pa x_2^4}=0$ is satisfied as the fourth-order derivatives are zero.

\item 
\be
\begin{array}{l}
\sigma_{11}=\frac{\pa^2  \phi}{\pa x_2^2}=2ax_1+6x_1x_2 \\
 \sigma_{22}=\frac{\pa^2  \phi}{\pa x_1^2}=0\\
\sigma_{12}=-\frac{\pa^2  \phi}{\pa x_1 \pa x_2}=-2ax_2-3x_2^2
\end{array}
\ee


\item At $x_2=c$, $\bs{n}=\bs{\hat{e}}_2$ and the traction vector $\bs{t}$ is:
\be
\bs{t}=\bs{\sigma} \bs{n}= \sigma_{12} \bs{\hat{e}}_1+\sigma_{22} \bs{\hat{e}}_2+\sigma_{23} \bs{\hat{e}}_3=\bs{0}
\ee
Hence we must have:
\be
\begin{array}{l}
\sigma_{12}(x_2=c)=0 \to -2ac-3c^2=0 \to a=-3c/2 \\
\sigma_{22}(x_2=c)=0 \to 0=0 \\
\sigma_{23}(x_2=c)=0  \quad \textrm{(always)}
\end{array}
\ee

\item On the remaining  boundaries we have:
\bi
\item At $x_1=0$, $\bs{n}=-\bs{\hat{e}}_1$, the boundary tractions are:
\be
\bs{\sigma}|_{x_1=0} \bs{n}=-\sigma_{11}(x_1=0)\bs{\hat{e}}_1-\sigma_{12}(x_1=0)\bs{\hat{e}}_2=(2ax_2+3x_2^2)\bs{\hat{e}}_2
\ee
\item At $x_1=b$, $\bs{n}=\bs{\hat{e}}_1$, the boundary tractions are:
\be
\bs{\sigma}|_{x_1=b} \bs{n}=\sigma_{11}(x_1=b)\bs{\hat{e}}_1+\sigma_{12}(x_1=b)\bs{\hat{e}}_2=(2ab+6bx_2)\bs{\hat{e}}_1-(2ax_2+3x_2^2)\bs{\hat{e}}_2
\ee
\item At $x_2=0$, $\bs{n}=-\bs{\hat{e}}_2$, the boundary tractions are:
\be
\bs{\sigma}|_{x_2=0} \bs{n}=
-\sigma_{12}(x_2=0)\bs{\hat{e}}_1-\sigma_{22}(x_2=0)\bs{\hat{e}}_2=0
\ee
\ei



\ei

\begin{figure}[!h]
\centering
\psfrag{x1}{$x_1$}
\psfrag{x2}{$x_2$}
\psfrag{h}{$h$}
\psfrag{l}{$l$}
\psfrag{1}{$1$}
\psfrag{p}{$P$}
\includegraphics[width=0.65\textwidth]{FIGURES/psbeam2.eps}
\label{fig:cantips}
\caption{Problem 4}
\end{figure}
%Lai 5.67
 \item Obtain the in-plane displacement components for the plane stress solution for the cantilever beam from the following strain-displacement relations:
\be
\epsilon_{11}=\frac{ \pa u_1}{\pa x_1}=-\frac{P x_1x_2}{E I}, \quad \epsilon_{22}=\frac{ \pa u_2}{\pa x_2}=\frac{\nu P x_1x_2}{E I}, \quad \epsilon_{12}=-\frac{P }{4 \mu  I} (h^2-x_2^2)
\ee

\underline{Solution:} \\

(This is the solved example 5.23.1 in Lai et al.). Given that the displacements $u_1,u_2$ are solely functions of $x_1,x_2$, from the strain-displacement equations, we obtain:
\be
\begin{array}{l}
 \epsilon_{11}=u_{1,1}=-\frac{P x_1x_2}{E I} \to u_1=-\frac{P x_1^2x_2}{2E I}+f_1(x_2) \\
 \epsilon_{22}=u_{2,2}=\frac{\nu P x_1x_2}{E I} \to u_2=\frac{\nu P x_1x_2^2}{2E I}+f_2(x_1) 
\end{array}
\ee
Finally:
\be
\begin{array}{l}
\epsilon_{12}= \frac{1}{2}(u_{1,2}+u_{2,1})=-\frac{P }{4 \mu  I} (h^2-x_2^2)  \\
\to   -\frac{P x_1^2}{2E I}+f_1'(x_2)+\frac{\nu P x_2^2}{2E I}+f_2'(x_1) =-\frac{P }{2 \mu  I} (h^2-x_2^2) \\
\to \frac{P x_1^2}{2E I}-f_2'(x_1)=f_1'(x_2)+\frac{\nu P x_2^2}{2E I}+\frac{P }{2 \mu  I} (h^2-x_2^2)
\end{array}
\ee
We observe that the term on the left is solely a function of $x_1$ and the term on the right only of $x_2$. The two terms can only be equal if they are both equal to a constant, say $b$. Hence:
\be
 \frac{P x_1^2}{2E I}-f_2'(x_1)=b \to f_2(x_1)=\frac{P x_1^3}{6E I} - bx_1 + c_2
 \ee
 and:
 \be
 f_1'(x_2)+\frac{\nu P x_2^2}{2E I}+\frac{P }{2 \mu  I} (h^2-x_2^2)=b \to f_1(x_2)=-\frac{\nu P x_2^3}{6E I}-\frac{P }{2 \mu 
 I}(h^2x_2-\frac{x_2^3}{3})+bx_2+c_1
 \ee
 As a result:
 \be
 \begin{array}{l}
 u_1=-\frac{P x_1^2x_2}{2E I}-\frac{\nu P x_2^3}{6E I}-\frac{P }{2 \mu  I}(h^2x_2-\frac{x_2^3}{3})+bx_2+c_1 \\
 u_2=\frac{\nu P x_1x_2^2}{2E I}-bx_1+\frac{P x_1^3}{6E I}+c_2
  \end{array}
\ee
 The coefficients $b,c_1,c_2$ should be determined from the boundary conditions.
 As in the example 5.23.1 in Lai et al., we require:
\be
u_{2,1}(l,0)=0, \quad u_1(l,0)=0, \quad u_2(l,0)=0.
\ee
These lead to:
\be
\begin{array}{l}
 b- \frac{P l^2}{2E I}=0 \to b=\frac{P l^2}{2E I} \\
 c_1=0 \\
 -bl+\frac{P l^3}{6E I}+c_2=0 \to c_2= \frac{Pl^3}{2E I} - \frac{Pl^3}{6EI} = \frac{Pl^3}{3EI}
\end{array}
\ee
We conclude that:
\be
 \begin{array}{l}
 u_1=-\frac{P x_1^2x_2}{2E I}-\frac{\nu P x_2^3}{6E I}-\frac{P }{2 \mu  I}(h^2x_2-\frac{x_2^3}{3})+\frac{P l^2}{2E I}x_2 \\
 u_2=\frac{\nu P x_1x_2^2}{2E I}+\frac{P x_1^3}{6E I}-\frac{P l^2}{2E I}x_1+\frac{P l^3}{3E I}
  \end{array}
\ee



\item (\textbf{Exam 2014}) 
The isotropic, elastic block ABC  ($E,\nu$ are given) in Figure \ref{fig:dam} is under plane strain conditions. It is subjected to hydrostatic pressure  along AB (as shown in the Figure) and is free along AC.
Given the Airy stress function:
\be
\phi(x,y)= \frac{1}{6} a_1 x^3+ \frac{1}{2} a_2 x^2 y+ \frac{1}{2} a_3x y^2 +\frac{1}{6}a_4 y^3
\ee
\bi
\item Determine  {\em all} the stress components  inside ABC using the traction boundary conditions above. You can ignore the block's own weight.
\item What tractions must be applied along BC in order for the block to be in equilibrium? Please provide a sketch with the appropriate values.

\ei
\begin{figure}[th]
\psfrag{l}{$l$}
\psfrag{p}{$\rho g y$}
\psfrag{a}{$A$}
\psfrag{B}{$B$}
\psfrag{C}{$C$}
\psfrag{x}{$x$}
\psfrag{y}{$y$}
\centering \includegraphics[width=0.45\textwidth,height=4cm]{FIGURES/dam.eps}
\caption{ Problem configuration}
\label{fig:dam}
\end{figure}
\newpage
\underline{Solution:} \\
\bi
\item 
% The biharmonic equation is satisfied since:
% \be
% \begin{array}{ll}
%  \nabla^4 \phi & =\frac{\pa^4  \phi}{\pa x^4}+2\frac{\pa^4  \phi}{\pa x^2 \pa y^2}+\frac{\pa^4  \phi}{\pa y^4} \\
%  & = 0 +0+0=0
% \end{array}
% \ee
The stress components are:
\be
\begin{array}{l}
\sigma_{xx}=\frac{\pa^2  \phi}{\pa y^2}=a_3x+a_4y\\
 \sigma_{yy}=\frac{\pa^2  \phi}{\pa x^2}=a_1x+a_2y\\
\sigma_{xy}=-\frac{\pa^2  \phi}{\pa x \pa y}=-(a_2x+a_3y)
\end{array}
\ee
Under plain strain conditions $\sigma_{xz}=\sigma_{yz}=0$ and $\sigma_{zz}=\nu (\sigma_{xx}+\sigma_{yy})$.

Along the boundary $AB$, the unit outward normal vector is $\bs{n}=-\hat{\bs{e}}_1$ and as a result:
\begin{align}
\bs t_{AB} &= \bs{\sigma}(x = 0,y) \bs{n}=-\sigma_{xx}(x=0,y) \hat{\bs{e}}_1-\sigma_{xy}(x=0,y) \hat{\bs{e}}_2 \\ 
&=-a_4y\hat{\bs{e}}_1+a_3 y \hat{\bs{e}}_2. \nonumber
\end{align}
The externally applied pressure gives rise to a traction vector $\bs{t}_{AB}=\rho g y \hat{\bs{e}}_1$. Hence:
\be
\begin{array}{c}
 -a_4y=\rho g y \to a_4=-\rho g, \\
 a_3 y =0 \to a_3=0.
\end{array}
\ee
Along the boundary $AC$, the unit outward normal vector is $\bs{n}=\frac{1}{\sqrt{2}}(\hat{\bs{e}}_1-\hat{\bs{e}}_2)$ and as a result:
\begin{align}
&\bs t_{AC} = \bs{\sigma}(x,y = x) \bs{n} \\ \nonumber 
&=\frac{1}{\sqrt{2}}\left( \left(\sigma_{xx}(x,y = x) -\sigma_{xy}(x,y = x)\right)\hat{\bs{e}}_1+(\sigma_{xy}(x,y = x)-\sigma_{yy}(x, y = x))\hat{\bs{e}}_2
\right)
\\
\nonumber
&=\frac{1}{\sqrt{2}}\left( (a_4x+a_2x)\hat{\bs{e}}_1+(-a_2x-(a_1+a_2)x)\hat{\bs{e}}_2 \right)  \\ \nonumber
&=\frac{1}{\sqrt{2}}\left( (a_4+a_2)x\hat{\bs{e}}_1-(a_1+2a_2)x\hat{\bs{e}}_2 \right).
\end{align}
The externally applied traction along $AC$ is $\bs{t}_{AC}=\bs{0}$. Hence:
\be
\begin{array}{c}
 (a_4+a_2)x=0 \to a_2=-a_4=\rho g, \\
 -(a_1+2a_2)x=0 \to a_1=-2a_2=-2\rho g.
\end{array}
\ee


\item Along $BC$, the unit outward normal vector is $\bs{n}=\hat{\bs{e}}_2$ and as a result the traction vector along this boundary is (Figure
\ref{fig:damsol}):
\be
\begin{array}{ll}
\bs{t}_{BC} & =\bs{\sigma}(x, y = l) \bs{n}=\sigma_{xy}(x, y = l) \hat{\bs{e}}_1+\sigma_{yy}(x, y = l) \hat{\bs{e}}_2 \\
& =-(a_2 x +a_3 l)\hat{\bs{e}}_1+(a_1x+a_2 l) \hat{\bs{e}}_2 \\
& = -\rho g x \hat{\bs{e}}_1+(-2\rho g x +\rho g l) \hat{\bs{e}}_2.
\end{array}
\ee


\begin{figure}[th]
\psfrag{l}{$l$}
\psfrag{p}{$\rho g y$}
\psfrag{a}{$A$}
\psfrag{B}{$B$}
\psfrag{C}{$C$}
\psfrag{x}{$x$}
\psfrag{y}{$y$}
\psfrag{tx1}{$-\rho g l$}
\psfrag{ty1}{$-\rho g l$}
\psfrag{ty2}{$\rho g l$}
\centering \includegraphics[width=0.45\textwidth,height=4cm]{FIGURES/dam_solution.eps}
\caption{ Solution}
\label{fig:damsol}
\end{figure}


\ei



\end{enumerate}

\end{document}
