\documentclass{article}
\usepackage{nips07submit_e,times}
%\documentstyle[nips07submit_09,times]{article}
\usepackage{subfigure}

\title{\underline{MSE - WS 2015/16}\\
Continuum Mechanics}


\author{
Instructor: P.S. Koutsourelakis  \\
\texttt{p.s.koutsourelakis@tum.de} \\
}

% The \author macro works with any number of authors. There are two commands
% used to separate the names and addresses of multiple authors: \And and \AND.
%
% Using \And between authors leaves it to \LaTeX{} to determine where to break
% the lines. Using \AND forces a linebreak at that point. So, if \LaTeX{}
% puts 3 of 4 authors names on the first line, and the last on the second
% line, try using \AND instead of \And before the third author name.


%
%%%%%%%%%%%%%%%%%%%%%%%%%%%%%%%%%%%%%%%%%%%%%%%%%%%%%%%
\usepackage{amsmath,amssymb}
\usepackage{hyperref} 

\usepackage[dvips]{psfrag,graphicx}


\linespread{1.6}
\newcommand{\ee}{\end{equation}}
\newcommand{\be}{\begin{equation}}
\newcommand{\ec}{\end{center}}
\newcommand{\bc}{\begin{center}}
\newcommand{\eea}{\end{eqnarray}}
\newcommand{\bea}{\begin{eqnarray}}
\newcommand{\bd}{\begin{description}}
\newcommand{\ed}{\end{description}}
\newcommand{\bi}{\begin{itemize}}
\newcommand{\ei}{\end{itemize}}
\newcommand{\pa}{\partial}
\newcommand{\bs}{\boldsymbol}
\def\RR{ \mathbb R}
\newcommand{\refeq}[1]{Equation (\ref{#1})}
%%%%%%%%%%%%%%%%%%%%%%%%%%%%%%%%%%%%%%%%%%%%%%%%%%%%%%
%

\begin{document}

\makeanontitle

%\begin{abstract}
%\input{abstract}
%\end{abstract}

\section*{\"Ubung06}




\begin{enumerate}

\item Consider a  bar of length $l$ and cross-sectional area $A$ as in Figure \ref{fig:bar1} and assume that there is a single non-zero displacement component
$u(x)$ (along the axial dimension).
The bar is loaded with a distributed force $f(x)=x$ (with units Force per length).  At $x=0$, $u(0)=0$.
Given the elastic modulus $E$, you are asked to:
\bi
\item Find the equation of equilibrium for the stress $\sigma(x)$ in the bar
\item Distinguish between essential and non-essential boundary conditions
\item Find the exact solution to this differential equation for the boundary conditions above
\item Write the potential energy functional
\item Amongst all displacement functions of the form $u(x)=a_0+a_1x+a_2x^2$ (i.e. $\forall a_0,a_1,a_2$), find the one that satisfies the essential boundary
conditions and minimizes the potential energy.
\item Consider a candidate solution of the form:
\be
u(x)=\left\{ \begin{array}{ll} a_1(1-\frac{2x}{l})+a_2 \frac{2x}{l}, & x\in [0,l/2] \\
                               a_3(2-\frac{2x}{l})+a_4 (\frac{2x}{l}-1)& x\in [l/2,l] 
             \end{array} \right.
\ee
and from the ones that satisfy the essential boundary conditions, find the minimizer of the potential energy. Compare the answer with the previous step.
\item Solve the problem by using the principle of virtual work and by employing candidate solutions as in the last two steps and admissible virtual
displacements of the same functional form.
\ei
\begin{figure}[th]
\subfigure[Problem 1]{
\psfrag{l}{$l$}
\psfrag{fx}{$f(x)=x$}
\includegraphics[width=0.45\textwidth,height=2cm]{FIGURES/bar1.eps}
\label{fig:bar1}
}
\hfill
\subfigure[Problem 2]{
\psfrag{l}{$l$}
\psfrag{F}{$F$}
\psfrag{k}{$k$}
\includegraphics[width=0.45\textwidth,height=2cm]{FIGURES/bar2.eps}
\label{fig:bar2}
}
\end{figure}


\underline{Solution:} \\

\bi
\item By considering the free-body diagram of a segment of length $\Delta x$ and taking the limit $\Delta x \to 0$, we obtain:
\be
A \frac{d \sigma}{dx}+f(x)=0
\ee
and from the stress-strain-displacement relation $\sigma(x)=E\epsilon(x)=E\frac{d u}{dx}$:
\be
EA \frac{d^2 u}{dx^2}+f(x)=0 
\ee
The general solution is:
\be
u(x)=\frac{1}{EA}(-x^3/6+c_1x+c_2)
\ee

\item The boundary conditions are:
\bi
\item $u(0)=0$ which is an {\em essential } boundary condition as it pertains to displacement specification
\item $A \sigma(l)=EA u'(l)=0 $ which is an {\em non-essential } boundary condition as it pertains to traction specification
\ei

\item From the boundary conditions we can obtain $c_1,c_2$. The exact solution is therefore:
\be
u(0)=0 \to c_2=0, \textrm{ and } EA u'(l)=0 \to c_1=l^2/2
\ee
and:
\be
u(x)=\frac{1}{EA}(-x^3/6+l^2x/2)
\ee

\item The general expression for the potential energy is $\Pi=U_{\textrm{int}}-W_{\textrm{ext}}$ where $U_{\textrm{int}}$ is the {\em strain energy}  and
$W_{\textrm{ext}}$ is the work potential of  the external forces:
\be
%01/02/2016: Edited by C. Grigo: added integration boundaries in 2. integral
\begin{array}{ll}
 \Pi&=\frac{1}{2} \int_0^l \sigma(x) \epsilon(x) A~dx-\int_0^l f(x) u(x) ~dx \\
 &=\frac{1}{2} EA \int_0^l (\frac{du}{dx})^2 ~dx-\int_0^l x u(x) ~dx \\
\end{array}
\ee


\item If $u(x)=a_0+a_1x+a_2x^2$ is to satisfy $u(0)=0$, then $a_0=0$. Hence the potential energy for functions of the form $u(x)=a_1x+a_2x^2$ becomes:
\be
\begin{array}{ll}
 \Pi &=\frac{1}{2} EA \int_0^l (a_1+2a_2x)^2 ~dx-\int_0^{l} x (a_1x+a_2x^2) ~dx \\
&= \frac{1}{2} EA (a_1^2l+2a_1a_2 l^2+4a_2^2 l^3/3)-a_1 l^3/3-a_2 l^4/4

\end{array}
\ee
This is a function of $a_1,a_2$. Hence minimizing $\Pi$ involves:
\be
\begin{array}{l}
 \frac{\pa \Pi}{\pa a_1}=0 \to \frac{1}{2} EA(2a_1l +2a_2l^2)-l^3/3=0 \\
  \frac{\pa \Pi}{\pa a_2}=0 \to \frac{1}{2} EA(2a_1l^2 +8a_2l^3/3)-l^4/4=0 
\end{array}
\label{eq:pe1}
\ee
the solution to the system above is $a_1=\frac{7l^2}{12EA}$, $a_2=-\frac{l}{4EA}$ and:
\be
u(x)=\frac{1}{EA}(\frac{7l^2}{12}x-\frac{l}{4}x^2)
\ee

The Hessian i.e. $\left[ \begin{array}{ll} \frac{\pa^2 \Pi}{\pa a_1^2} & \frac{\pa^2 \Pi}{\pa a_1a_2} \\ \frac{\pa^2 \Pi}{\pa a_1a_2} & \frac{\pa^2 \Pi}{\pa a_2^2} \end{array} \right]=EA \left[ \begin{array}{ll}  l  & l^2 \\ l^2 &  8l^3/3 \end{array} \right]$ is constant and positive definite. Hence the solution found is a global minimizer of the potential energy.


\item For the piecewise-linear candidate solution of the form:
\be
u(x)=\left\{ \begin{array}{ll} a_1(1-\frac{2x}{l})+a_2 \frac{2x}{l}, & x\in [0,l/2] \\
                               a_3(2-\frac{2x}{l})+a_4 (\frac{2x}{l}-1)& x\in [l/2,l] 
             \end{array} \right.
\ee
we must have $a_1=0$ in order for $u(0)=0$. Furthermore in order to ensure continuity at $x=l/2$, $a_2=a_3$. Hence we consider functions:
\be
u(x)=\left\{ \begin{array}{ll} a_2 \frac{2x}{l}, & x\in [0,l/2] \\
                               a_2(2-\frac{2x}{l})+a_4 (\frac{2x}{l}-1)& x\in [l/2,l] 
             \end{array} \right.
\ee
             Substituting in the potential energy function, we get:
\be
\begin{array}{ll}
 \Pi &=\frac{1}{2} EA (\int_0^{l/2} (2a_2/l)^2 ~dx+\frac{1}{2} EA\int_{l/2}^l (-2a_2/l+2a_4/l)^2 ~dx \\
 &-\int_0^{l/2} x a_2 \frac{2x}{l} ~dx -\int_{l/2}^l x (a_2(2-\frac{2x}{l})+a_4 (\frac{2x}{l}-1) ) ~dx \\
&= \frac{1}{2} EA (2a_2^2/l+2\frac{(a_2-a_4)^2}{l})-\frac{a_2l^2}{12}-\frac{1}{24} l^2(4a_2+5a_4)
\end{array}
 \ee            
 This is a function of $a_2,a_4$. Hence minimizing $\Pi$ involves:
\be
\begin{array}{l}
 \frac{\pa \Pi}{\pa a_2}=0 \to \frac{1}{2} EA(4a_2/l +4(a_2-a_4)/l)-l^2/12-l^2/6=0 \\
  \frac{\pa \Pi}{\pa a_4}=0 \to \frac{1}{2} EA(4(a_4-a_2)/l)- \frac{5l^2}{24} =0 \\
\end{array}
\label{eq:pe2}
\ee
the solution to the system above is $a_2=\frac{11 l^3}{48EA}$, $a_4=\frac{l^3}{3EA}$.

It can also be verified that the Hessian i.e. $\left[ \begin{array}{ll} \frac{\pa^2 \Pi}{\pa a_1^2} & \frac{\pa^2 \Pi}{\pa a_1a_2} \\ \frac{\pa^2 \Pi}{\pa a_1a_2} & \frac{\pa^2 \Pi}{\pa a_2^2} \end{array} \right]$ is constant and positive definite. Hence the solution found is a global minimizer of the potential energy.
Figure \ref{fig:sol} compares the exact solution with the approximate ones obtained by employing the principle of minimum potential energy.
\vspace{.4cm}
\begin{figure}[!h]
\centering
\psfrag{x}{$x$}
\psfrag{u}{$u(x)$}
\includegraphics[height=4cm,width=.6\textwidth]{FIGURES/potential.eps}
\label{fig:sol}
\caption{Comparison of potential energy solutions for problem 1}
\end{figure}
 
 
 %
 \item Admissible virtual displacements $\delta u$ should be zero wherever essential boundary conditions are prescribed.
 In the first case, if $\delta u(x)=b_0+b_1x+b_2x^2$ then we must have $b_0=0$ i.e. $\delta u(x)=b_1x+b_2x^2$. The corresponding virtual strain is $\delta \epsilon(x)=\frac{d \delta u(x)}{dx}=b_1+2b_2x$. For candidate solutions of the form $u(x)=a_1x+a_2x^2$, the principle of virtual work becomes:
 \be
 \delta W_{\textrm{int}}=\delta W_{\textrm{ext}} \to \int_{0}^l \sigma(x) \delta \epsilon (x) ~dV=\int_0^l f(x) \delta u(x) ~dx
 \ee
 and:
 \be
 EA \int_{0}^l \frac{d u(x)}{dx} \frac{d \delta u(x)}{dx} ~dx =\int_0^l x ~\delta u(x) ~dx
 \label{eq:pvw}
 \ee
 This should hold for every virtual displacement $\delta u$. Since these are parametrized by $b_1,b_2$, it should hold for all $b_1,b_2$. In particular:
 \be
 \begin{array}{l}
 EA \int_{0}^l (a_1+2a_2x)(b_1+2b_2x) ~dx =\int_0^l x (b_1x+b_2x^2)~dx, \quad \forall b_1,b_2 \\
  EA(a_1b_1 l+a_1b_2 l^2+a_2b_1 l^2+\frac{4}{3} a_2b_2 l^3)=b_1\frac{l^3}{3}+b+2\frac{l^4}{4}, \quad \forall b_1,b_2 \\
  b_1 \left(EA l a_1+EA a_2 l^2-\frac{l^3}{3} \right)  +b_2 \left(EAa_1l^2+EA \frac{4}{3} a_2 l^3-\frac{l^4}{4} \right), \quad \forall b_1,b_2 \\
  \end{array}
 \ee
 If this is to hold $\forall b_1,b_2$, then:
 
\be
\begin{array}{l}
 EAa_1 l+EA a_2 l^2-\frac{l^3}{3}=0 \\
 EAa_1l^2+EA \frac{4}{3} a_2 l^3-\frac{l^4}{4}=0
\end{array}
\ee
We observe that these are the same equations for $a_1,a_2$ that we obtained with the principle of minimum potential energy (\refeq{eq:pe1}).
Hence the  solution to the system above is again  $a_1=\frac{7l^2}{12EA}$, $a_2=-\frac{l}{4EA}$ and:
\be
u(x)=\frac{1}{EA}(\frac{7l^2}{12}x-\frac{l}{4}x^2)
\ee


\item For piecewise linear candidate solutions of the form:
\be
u(x)=\left\{ \begin{array}{ll} a_2 \frac{2x}{l}, & x\in [0,l/2] \\
                               a_2(2-\frac{2x}{l})+a_4 (\frac{2x}{l}-1)& x\in [l/2,l] 
             \end{array} \right.
\ee
we use virtual displacements $\delta u(x)$ of the form:
\be
\delta u(x)=\left\{ \begin{array}{ll} b_2 \frac{2x}{l}, & x\in [0,l/2] \\
                               b_2(2-\frac{2x}{l})+b_4 (\frac{2x}{l}-1)& x\in [l/2,l] 
             \end{array} \right.
\ee
which are admissible i.e. $\delta u(0)=0$ and continuous.
By substituting in the principle of virtual work (\refeq{eq:pvw}) we arrive to 2 equations for $a_2,a_4$ which are identical to \refeq{eq:pe2} and therefore lead to the same approximate solution as the principle of minimum potential energy.
\ei


\item Consider a  bar of length $l$ and cross-sectional area $A$ as in Figure \ref{fig:bar2} and assume that there is a single non-zero displacement component $u(x)$ (along the axial dimension).
The bar is loaded with a distributed force $F$ at $x=0$ and is connected with a linear spring, with spring constant $k$, at $x=l$. The other end of the  spring is fixed.
Given the elastic modulus $E$, you are asked to:
\bi
\item Find the equation of equilibrium for the stress $\sigma(x)$ in the bar
\item Find the exact solution to this differential equation for the boundary conditions above
\item Write the potential energy functional
\item Amongst all displacement functions of the form $u(x)=a_0+a_1x$ (i.e. $\forall a_0,a_1$), find the one that  minimizes the potential energy.
\item Solve the problem by using the principle of virtual work and by employing candidate solutions as in the last  step.
\ei

\underline{Solution:} \\
\bi
\item By considering the free-body diagram of a segment of length $\Delta x$ and taking the limit $\Delta x \to 0$, we obtain:
\be
A \frac{d \sigma}{dx}=0
\ee
and from the stress-strain-displacement relation $\sigma(x)=E\epsilon(x)=E\frac{d u}{dx}$:
\be
EA \frac{d^2 u}{dx^2}=0 
\ee
The general solution is:
\be
u(x)=c_1x+c_2
\ee

\item The boundary conditions are:
\bi
\item at $x=0$: $A\sigma(0)+F=0 \to EAu'(0)=-F$ 
\item at $x=l$ a spring force $F_s$ of magnitude $ku(l)$ develops and $A \sigma(l)+F_s=0 \to EA u'(l)+ku(l)=0 $
\ei

\item From the boundary conditions we can obtain $c_1=-\frac{F}{EA}$ and $c_2=F(\frac{1}{k}+\frac{l}{EA})$. The exact solution is therefore:
\be
u(x)=-\frac{F}{EA}x+F(\frac{1}{k}+\frac{l}{EA})
\ee


\item The general expression for the potential energy is $\Pi=U_{\textrm{int}}-W_{\textrm{ext}}$ where $U_{\textrm{int}}$ is the {\em  strain energy}  and
$W_{\textrm{ext}}$ is the work potential of the external forces. We can consider the spring as part of the system (easier) in which case its contribution will
appear in $U_{\textrm{int}}$, or we can consider the spring as external to the system in which case its contribution will appear in $W_{\textrm{ext}}$ through
the spring force $F_s$. Both considerations should lead to the same result:
\be
\begin{array}{ll}
 \Pi&=\frac{1}{2} \int_0^l \sigma(x) \epsilon(x) A~dx+\frac{1}{2} ku^2(l)-F u(0) \\
 &=\frac{1}{2} EA \int_0^l (\frac{du}{dx})^2 ~dx+\frac{1}{2} ku^2(l)-F u(0)\\
\end{array}
\ee


\item For $u(x)=a_0+a_1x$, the potential energy becomes:
\be
\begin{array}{ll}
 \Pi &=\frac{1}{2} EA \int_0^l (\frac{du}{dx})^2 ~dx+\frac{1}{2} ku^2(l)-F u(0)\\
 &= \frac{1}{2} EA \int_0^l (a_1)^2 ~dx+\frac{1}{2} k(a_0+a_1l)^2-F a_0\\
 &= \frac{1}{2} EA  a_1^2 l+\frac{1}{2} k(a_0+a_1l)^2-F a_0\\
\end{array}
\ee

This is a function of $a_0,a_1$. Hence minimizing $\Pi$ involves:
\be
\begin{array}{l}
 \frac{\pa \Pi}{\pa a_0}=0 \to  k(a_0+a_1l)-F=0 \\
  \frac{\pa \Pi}{\pa a_2}=0 \to  EA a_1 l + k(a_0+a_1l)l=0
\end{array}
\ee
The solution to the system above is $a_1=-\frac{F}{EA}$ and $a_0=F(\frac{1}{k}+\frac{l}{EA})$ which coincides with the exact solution.

\item With regards to the principle of virtual work $\delta W_{\textrm{int}}=\delta W_{\textrm{ext}}$ there are again two options. We can consider the spring as part of
the system in which case its contribution will appear in $\delta W_{\textrm{int}}$ through the work done by the internal spring force. Alternatively, we can
consider the spring as external to the system (easier) in which case its contribution will appear in $\delta W_{\textrm{ext}}$ through the work done by $F_s$ when a
virtual displacement $\delta u(l)$ is imposed.
For virtual displacements $\delta u(x)=b_0+b_1x$ we obtain:
\be
\begin{array}{l}
\delta W_{\textrm{int}}=\delta W_{\textrm{ext}}  \\
\to EA \int_{0}^l \frac{d u(x)}{dx} \frac{d \delta u(x)}{dx} ~dx=-F_s \delta u(l)+F\delta u(0) \\
\to EA \int_{0}^l a_1 b_1=-k(a_0+a_1l)(b_0+b_1l)+Fb_0 \\
\to EA a_1 b_1 l=-k(a_0b_0+a_0b_1l+a_1b+0l+a_1b_1l^2)+Fb_0 \\
\to b_0(ka_0+ka_1l-F)+b_1(EAa_1l+ka_0l+a_1l^2)=0
\end{array}
\ee
This should hold for all virtual displacements, i.e. $\forall b_0,b_1$ which leads to two equations:
\be
\begin{array}{l}
  k(a_0+a_1l)-F=0 \\
    EA a_1 l + k(a_0+a_1l)l=0
\end{array}
\ee
which are the same as in the principle of minimum potential energy. Hence we obtain again the exact solution.

\ei








\item Consider the beam as in Figure \ref{fig:cantips} with known material properties $E,\nu$,  load $P$, and which is assumed to be under plane stress conditions.
Furthermore we assume that $\sigma_{22}=0$.
The following boundary conditions are assumed at $(l,0)$: $u_1=0$, $u_2=0$ and $\frac{\pa u_2}{\pa x_1}=0$. 
The non-zero displacement fields $u_1, u_2$ are assumed to be of the form:
\be
u_1(x_1,x_2)=a_1x_1^2x_2 +a_2 x_2^3+a_3x_2+a_4, \quad u_2(x_1,x_2)=\frac{\nu P}{2EI} x_1x_2^2+\frac{P}{6EI} x_1^3+a_5x_1+a_6
\ee
where $a_1, a_2, a_3, a_4, a_5, a_6$ are real-valued parameters.
\bi
\item Write the expression for the potential energy of this beam
\item Amongst all displacement functions $u_1,u_2$ above that satisfy the three boundary conditions at $(l,0)$, find the one that minimizes the potential energy and compare with the result in Problem 4 of \"Ubung05.
\ei

\begin{figure}[!h]
\centering
\psfrag{x1}{$x_1$}
\psfrag{x2}{$x_2$}
\psfrag{h}{$h$}
\psfrag{l}{$l$}
\psfrag{1}{$1$}
\psfrag{p}{$P$}
\includegraphics[height=3cm]{FIGURES/psbeam2.eps}
\label{fig:cantips}
\caption{Problem 3}
\end{figure}

\underline{Solution:} \\
\bi
\item By imposing the boundary conditions we obtain $a_4=0, a_5=-\frac{Pl^2}{2EI}$ and $a_6=\frac{Pl^3}{3EI}$. Hence:
\be
u_1(x_1,x_2)=a_1x_1^2x_2 +a_2 x_2^3+a_3x_2, \quad u_2(x_1,x_2)=\frac{\nu P}{2EI} x_1x_2^2+\frac{P}{6EI} x_1^3-\frac{Pl^2}{2EI}x_1+\frac{Pl^3}{3EI}
\ee
From the strain-displacement equations we obtain:

\be
\begin{array}{l}
 \epsilon_{11}=u_{1,1}=2a_1x_1x_2 \\
\epsilon_{22}=u_{2,2}=\frac{\nu P}{EI} x_1x_2 \\
\epsilon_{12}=\frac{1}{2}(a_1x_1^2+3a_2x_2^2+a_3+\frac{\nu P}{2EI}  x_2^2+\frac{P}{2EI} x_1^2-\frac{Pl^2}{2EI} )
\end{array}
\ee
We also note that under plane stress $\sigma_{33}=0 \to \epsilon_{33}=-\frac{\lambda}{\lambda+2\mu}(\epsilon_{11}+\epsilon_{22})$.
The corresponding stresses, can be found from the stress-strain relationship:
\be
\begin{array}{l}
 \sigma_{11}=(\lambda+2\mu)\epsilon_{11}+\lambda(\epsilon_{22}+\epsilon_{33})\\
  \sigma_{22}=(\lambda+2\mu)\epsilon_{22}+\lambda(\epsilon_{11}+\epsilon_{33})\\
\sigma_{12}=2\mu \epsilon_{12}=\mu (a_1x_1^2+3a_2x_2^2+a_3+\frac{\nu P}{2EI}  x_2^2+\frac{P}{2EI} x_1^2-\frac{Pl^2}{2EI} )
\end{array}
\ee
Given that $\sigma_{22}=0$ we obtain:
\be
\begin{array}{ll}
 0 &= (\lambda+2\mu)\epsilon_{22}+\lambda(\epsilon_{11}+\epsilon_{33}) \\ 
 0 &= (\lambda+2\mu)\epsilon_{22}+\lambda\epsilon_{11}-\frac{\lambda^2}{\lambda+2\mu}(\epsilon_{11}+\epsilon_{22}) \\
 0 & = 4\mu \frac{\lambda+\mu}{\lambda+2\mu} \epsilon_{22}+\frac{2\mu \lambda}{\lambda+2\mu} \epsilon_{11} \\
 0 & = 4\mu \frac{\lambda+\mu}{\lambda+2\mu} \frac{\nu P}{EI} x_1x_2 +\frac{2\mu \lambda}{\lambda+2\mu} 2a_1x_1x_2  \\
 0 & = \frac{\lambda+\mu}{\lambda+2\mu} \frac{\nu P}{EI} +\frac{ \lambda}{\lambda+2\mu} a_1 \\
 a_1 & = - \frac{\lambda+\mu}{\lambda }\frac{\nu P}{EI}
%  \to 0 & =(\lambda+2\mu)\frac{\nu P}{EI} x_1x_2+\lambda(2a_1x_1x_2 ) \\
%  \to a_1 &=-\frac{\lambda+2\mu}{2\lambda} \frac{\nu P}{EI}
\end{array}
\label{eq31}
\ee
and since $\nu=\frac{\lambda}{2(\lambda+\mu)}$ we have that $a_1=-\frac{ P}{2EI}$.
We also note that under plane stress conditions $\sigma_{13}=\sigma_{23}=\sigma_{33}=0$.
The general expression for the potential energy involves the internal energy $U_{\textrm{int}}$ which is given by:
\be
U_{\textrm{int}}=\frac{1}{2} \int \sigma_{ij}\epsilon_{ij}~dV
\ee
In the plane stress conditions considered and accounting for the zero stress components,  this becomes:
\be
U_{\textrm{int}}=\frac{1}{2} \int (\sigma_{11}\epsilon_{11}+\sigma_{22}\epsilon_{22}+2\sigma_{12}\epsilon_{12})~dV
\ee
As for the $W_{\textrm{ext}}$  this is:
\be
W_{\textrm{ext}}= \int_{-h}^h 1~ \frac{P}{2h} u_2(0,x_2)~dx_2= \frac{ P^2 l^3}{3EI}
\ee
Hence the potential energy is:
\be
\begin{array}{ll}
 \Pi &= U_{\textrm{int}}-W_{\textrm{ext}} \\
 &= \frac{1}{2} \int (\sigma_{11}\epsilon_{11}+\sigma_{22}\epsilon_{22}+2\sigma_{12}\epsilon_{12})~dV-\frac{P^2 l^3}{3EI} \\
 &=\frac{1}{2} \int_0^l \int_{-h}^h  (\sigma_{11}\epsilon_{11}+\sigma_{22}\epsilon_{22}+2\sigma_{12}\epsilon_{12})~dx_1 dx_2-\frac{P^2 l^3}{3EI} \\
 &=\frac{1}{2} \int_0^l \int_{-h}^h  (\sigma_{11}\epsilon_{11}+2\sigma_{12}\epsilon_{12})~dx_1 dx_2-\frac{P^2 l^3}{3EI} \\
%  & = \frac{1}{2} \int_0^l \int_{-h}^h ( (\lambda+2\mu)\epsilon_{11}^2+\lambda\epsilon_{22} \epsilon_{11}+4\mu \epsilon_{12}^2) ~dx_1 dx_2-\frac{2h P^2 l^3}{3EI} \\
 \end{array}
\ee

By substituting the expressions above, the potential energy will eventually become a function of $a_2,a_3$. Minimization of $\Pi$ involves setting $\frac{\pa \Pi}{\pa a_i}=0, i=2,3$ which leads to a system of 2 equations with 2 unknowns.
The solution found should be identical to the exact solution found in Problem 4 of \"Ubung05.
\ei


\item(\textbf{Exam 2014}) 
Consider the  material body occupying the rectangle with area $A=(2L) \cdot (2c) =4 Lc$ shown in Figure  3 %\ref{fig:pvw}
which is under plane stress in the $x_1-x_2$ plane. Assume it has a unit thickness (in the $x_3$ direction) and is subjected to the equal and opposite normal tractions $q(x_2)=q_0 x_2^2$ (Force per unit area) shown.
\bi
\item Using the Principle of Virtual Work and a virtual displacement field $\delta u_1= x_1$ , $\delta u_2=\delta u_3=0$, find  the average value of $\sigma_{11}$ over the rectangle i.e. :
\be
\frac{1}{A} \int \sigma_{11}(x_1,x_2) dA 
\ee
\ei
\begin{figure}[!h]
\psfrag{q}{$q(x_2)$}
\psfrag{L}{$L$}
\psfrag{c}{$c$}
\psfrag{x1}{$x_1$}
\psfrag{x2}{$x_2$}
\centering
  \includegraphics[height=5cm]{FIGURES/pvw.eps}
 \caption{Problem Configuration}
 \label{fig:pvw}
\end{figure}




\underline{Solution:} \\

One must first find the virtual strains $\delta \epsilon_{ij} =\frac{1}{2} (\delta u_{i,j}+\delta u_{j,i})$ corresponding to the virtual displacement field considered:
\be
\begin{array}{l}
 \delta \epsilon_{11} = \frac{ \pa \delta u_1}{\pa x_1}=1 \\
 \delta \epsilon_{22} =\delta \epsilon_{33} =\delta \epsilon_{12} =\delta \epsilon_{13} =\delta \epsilon_{23} =0
\end{array}
\ee
The internal virtual work $\delta W_{\textrm{int}}$ is then given by:
\be
\begin{array}{ll}
 \delta W_{\textrm{int}} & = \int \sigma_{ij} \delta \epsilon_{ij}~dV \\
  & = 1\int \sigma_{11} \delta \epsilon_{11} dA \qquad \textrm{(unit~thickness)} \\
  & = \int \sigma_{11} ~dA \\
\end{array}
\ee
The external virtual work $\delta W_{\textrm{ext}}$ is then given by:
\be
\begin{array}{ll}
 \delta W_{\textrm{ext}} & = \underbrace{\int q(x_2) \delta u_1 (x_1=L) dA}_{\textrm{right~boundary}} - \underbrace{\int q(x_2) \delta u_1 (x_1=-L)
 dA}_{\textrm{left~boundary}} \\
 &= L\int_{-c}^c q(x_2) dx_2 + L\int_{-c}^c q(x_2) dx_2 \qquad \textrm{(unit~thickness)} \\ 
 & = 2 L \int_{ -c}^{c} q(x_2) ~dx_2 \\
 & = 2 L \int_{ -c}^{c} q_0 x_2^2 ~dx_2 \\ 
 & = 2L q_0 2 \frac{c^3}{3} =4Lq_0 \frac{c^3}{3}
 \end{array}
 \ee
 From the equality $\delta W_{\textrm{int}}=\delta W_{\textrm{ext}}$ we have that:
 \be
 \int \sigma_{11} ~dA =\frac{4}{3} Lq_0 c^3 \quad \to \quad \frac{1}{A} \int \sigma_{11} ~dA =q_0 \frac{c^2}{3}.
 \ee
 
 
\end{enumerate}

\end{document}
